% This is a Latex tutorial document

\documentclass{article}
% \usepackage[letterpaper, landscape, margin=2in]{geometry}
\usepackage[margin=1.0in]{geometry}
\usepackage{amsmath}
\usepackage{amssymb}
\usepackage{empheq}
\usepackage{mathtools}
\usepackage{graphicx}
\usepackage{pgfplots}
\usepackage{cancel}
\usepackage{enumitem}
\graphicspath{ {./../../../Documents/GradSchool/MECE6374/} }
\makeatletter
\def\@seccntformat#1{%
  \expandafter\ifx\csname c@#1\endcsname\c@section\else
  \csname the#1\endcsname\quad
  \fi}
\makeatother
\title{MECE 6374: Fun Work \#2}
\date{\today}
\author{Eric Eldridge (1561585)}
\begin{document}
  \maketitle

  \section{Problem 1}
  Consider the ODE model
  \begin{equation*}
    \ddot{y}+\dot{y}+y-\frac{1}{16}y^5 = 0 \tag{1.1}
  \end{equation*}
  
  \begin{enumerate}[label=(\roman*)]
    \item Write the model in a state space form $\dot{x}=f(x)$ and compute the
      equilibrium points.
    \item Use linearization to determine the local stability properties of
        each equilibrium point.
  \end{enumerate}
      
  \noindent \textit{Solution} \newline

  i\big) Let $x_1=y$, $x_2=\dot{y}$. Then 
  \begin{align*}
    \dot{x_1} &= x_2 \tag{1.2} \\
    \dot{x_2} &+ x_2 + x_1 -\frac{1}{16}x_1^5 = 0 \\
    \dot{x_2} &= \frac{1}{16}x_1^5 - x_1 - x_2 \tag{1.3} \\
    \begin{bmatrix}
      \dot{x_1} \\
      \dot{x_2}
    \end{bmatrix} &=
    \begin{bmatrix}
      x_2 \\
      \frac{1}{16}x_1^5 - x_1 - x_2
    \end{bmatrix} 
  \end{align*} \newline

  The equilibrium points $(\bar{x}_1, \bar{x}_2)$ are found by setting $\dot{x_1}=\dot{x_2}=0$.

  \begin{align*}
    \dot{x_1} = x_2 = 0 \implies \bar{x}_2=0 \tag{1.4} \\
    \dot{x_2} = \frac{1}{16}x_1^5 - x_1 - x_2 = 0 \tag{1.5} \\
  \end{align*}

  Plugging in [1.4] into [1.5] gives us

  \begin{align*}
    \frac{1}{16}x_1^5 - x_1 &= 0 \\
    x_1(\frac{1}{16}x_1^4 - 1) &= 0 \\
    \bar{x}_1 &= 0, \pm\hspace{0.75mm}2, \pm\hspace{0.75mm}2j \tag{1.6} \\
    \bar{x}_2 &= 0 \tag{1.7} 
  \end{align*}

  \begin{empheq}[box=\fbox]{align}
    \nonumber \text{Equilibrium Points}: (0,0), (0,2), (0,-2)
  \end{empheq} \newline

  \newpage

  ii\big) $\dot{x_1}=x_2$ is already linearized so we only need to linearize
  $\dot{x_2}=f_2(x)=\frac{1}{16}x_1^5 - x_1 - x_2$. 

  The formula for linearization is
  \begin{align*}
    f_{2, lin}(x_1,x_2) &= \cancelto{0}{f_2(\bar{x}_1, \bar{x}_2)} \hspace{3.0mm} + \frac{\partial f_2}{\partial x_1}\bigg|_{(\bar{x}_1, \bar{x}_2)}(x_1-\bar{x}_1) + \frac{\partial f_2}{\partial x_2}\bigg|_{(\bar{x}_1, \bar{x}_2)}(x_2-\bar{x}_2) + \cancelto{0}{H.O.T.} \tag{1.8} \\
    f_{2, lin}(x_1,x_2) &= (\frac{5}{16}x_1^4 -1)\bigg|_{(x_1,x_2)}(x_1-\bar{x}_1) - x_2 
  \end{align*}

  \textbf{\underline{Equilibrium Point: (0,\hspace{0.8mm}0)}}

  \begin{align*}
    f_{1, lin}(x_1,x_2) &= x_2 \\
    f_{2, lin}(x_1,x_2) &= (\frac{5}{16}x_1^4 -1)\bigg|_{(0,0)}x_1 - x_2 \\
    f_{2, lin}(x_1,x_2) &= -x_1 -x_2 \\
    \begin{bmatrix}
      \dot{x_1} \\
      \dot{x_2}
    \end{bmatrix} &=
    \begin{bmatrix}
      x_2 \\
      - x_1 - x_2
    \end{bmatrix} \\
    \begin{bmatrix}
      \dot{x_1} \\
      \dot{x_2}
    \end{bmatrix} &=
    \begin{bmatrix}
      0 & 1 \\
      -1&  -1
    \end{bmatrix} 
    \begin{bmatrix}
      x_1 \\
      x_2
    \end{bmatrix} 
  \end{align*}

  The eigenvalues of the matrix $\begin{bmatrix}
      0 & 1 \\
      -1&  -1
  \end{bmatrix}$ are $\lambda_{1,2} = -\frac{1}{2} \pm \frac{\sqrt{3}}{2}j$
  \newline
  \hangindent=1.58em
  \hangafter=1 Because both of the eignevalues have negative real parts, we know that \textbf{this
  equilibrium point is locally stable.} \newline
  
  \textbf{\underline{Equilibrium Point: (0,\hspace{0.8mm}2)}}

  \begin{align*}
    f_{1, lin}(x_1,x_2) &= x_2 \\
    f_{2, lin}(x_1,x_2) &= (\frac{5}{16}x_1^4 -1)\bigg|_{(0,2)}(x_1-2) - x_2 \\
    f_{2, lin}(x_1,x_2) &= 4(x_1-2) -x_2 \\
  \end{align*}
  \indent Now we will define $\tilde{x}_1, \tilde{x}_2$, s.t.
  \begin{align*}
    \tilde{x}_1 &= x_1-2 \\
    \tilde{x}_2 &= x_2 \\
    \implies \dot{\tilde{x}}_1 &= \dot{x}_1 \\
    \implies \dot{\tilde{x}}_2 &= \dot{x}_2
  \end{align*}
  
  \indent Plugging in gives us
  \begin{align*}
    \begin{bmatrix}
      \dot{\tilde{x}}_1 \\
      \dot{\tilde{x}}_2
    \end{bmatrix} &=
    \begin{bmatrix}
      \tilde{x}_2 \\
      4\tilde{x}_1 - \tilde{x}_2
    \end{bmatrix} \\
    \begin{bmatrix}
      \dot{\tilde{x}}_1 \\
      \dot{\tilde{x}}_2
    \end{bmatrix} &=
    \begin{bmatrix}
      0 & 1 \\
      4 & -1
    \end{bmatrix} 
    \begin{bmatrix}
      \tilde{x}_1 \\
      \tilde{x}_2
    \end{bmatrix} 
  \end{align*}

  The eigenvalues of the matrix $\begin{bmatrix}
      0 & 1 \\
      4&  -1
  \end{bmatrix}$ are $\lambda_{1,2} = -2.56,\hspace{0.8mm}1.56$ \newline \newline
  
  \hangindent=1.58em
  \hangafter=1 Because one of the eignevalues has negative real part and one of
  the eigenvalues has positive real part, we know that \textbf{this
    equilibrium point is a saddle point.}
  \newpage
  
  \textbf{\underline{Equilibrium Point: (0,\hspace{0.8mm}-2)}}

  \begin{align*}
    f_{1, lin}(x_1,x_2) &= x_2 \\
    f_{2, lin}(x_1,x_2) &= (\frac{5}{16}x_1^4 -1)\bigg|_{(0,-2)}(x_1+2) - x_2 \\
    f_{2, lin}(x_1,x_2) &= 4(x_1+2) -x_2 \\
  \end{align*}
  \indent Now we will define $\tilde{x}_1, \tilde{x}_2$, s.t.
  \begin{align*}
    \tilde{x}_1 &= x_1+2 \\
    \tilde{x}_2 &= x_2 \\
    \implies \dot{\tilde{x}}_1 &= \dot{x}_1 \\
    \implies \dot{\tilde{x}}_2 &= \dot{x}_2
  \end{align*}
  
  \indent Plugging in gives us
  \begin{align*}
    \begin{bmatrix}
      \dot{\tilde{x}}_1 \\
      \dot{\tilde{x}}_2
    \end{bmatrix} &=
    \begin{bmatrix}
      \tilde{x}_2 \\
      4\tilde{x}_1 - \tilde{x}_2
    \end{bmatrix} \\
    \begin{bmatrix}
      \dot{\tilde{x}}_1 \\
      \dot{\tilde{x}}_2
    \end{bmatrix} &=
    \begin{bmatrix}
      0 & 1 \\
      4 & -1
    \end{bmatrix} 
    \begin{bmatrix}
      \tilde{x}_1 \\
      \tilde{x}_2
    \end{bmatrix} 
  \end{align*}

  The eigenvalues of the matrix $\begin{bmatrix}
      0 & 1 \\
      4&  -1
  \end{bmatrix}$ are $\lambda_{1,2} = -2.56,\hspace{0.8mm}1.56$ \newline \newline
  
  \hangindent=1.58em
  \hangafter=1 Because one of the eignevalues has negative real part and one of
  the eigenvalues has positive real part, we know that \textbf{this
    equilibrium point is a saddle point.}
  \newpage  

  \section{Problem 2}

  Consider the following system

  \begin{align*}
    \dot{x_1} &= x_1 - x_2^2 \tag{2.1} \\
    \dot{x_2} &= 6x_2 + x_1^2 - 7x_2^2 \tag{2.2} 
  \end{align*}

  a) Find all equilibrium points \newline
  \hangindent=1.58em
  \hangafter=1 b) Use linearization to determine the local stability and type of each equilibrium point and sketch the approximate phase portrait near each point. \newline
  
  \noindent \textit{Solution} \newline
  
  a) The equilibrium points are found by setting $\dot{x_1}=\dot{x_2}=0$.

  \begin{align*}
    \dot{x_1} &= x_1 - x_2^2 = 0 \implies x_1 = x_2^2 \tag{2.3} \\
    \dot{x_2} &= 6x_2 + x_1^2 - 7x_2^2 = 0 \tag{2.4} 
  \end{align*}

  Plugging in [2.3] into [2.4] gives us

  \begin{align*}
    6x_2 + (x_2^2)^2 - 7x_2^2 = 0 \\
    x_2^4 - 7x_2^2 + 6x_2 = 0 \\
    x_2(x_2^3 - 7x_2 + 6) = 0 \\
    x_2(x_2-2)(x_2-1)(x_2+3) = 0 \\
  \end{align*}
  % \begin{empheq}[box=\fbox]{align}
  %     \nonumber \bar{x}_2 &= -3,0,1,2  \tag{mytag} \\
  %     \nonumber \bar{x}_1 &= 9,0,1,4 \tag{other tag}
  % \end{empheq} \newline

  \begin{align*}
    \bar{x}_2 &= -3,0,1,2 \\
    \bar{x}_1 &= 9,0,1,4
  \end{align*}
  
  \begin{empheq}[box=\fbox]{align}
    \nonumber \text{Equilibrium Points}: (9,-3), (0,0), (1,1), (4,2) \tag{2.5}
  \end{empheq} \newline

  b) As stated in [1.8], the formula for linearization is
  \begin{align*}
    f_{i, lin}(x_1,x_2) &= \frac{\partial f_i}{\partial x_1}\bigg|_{(\bar{x}_1, \bar{x}_2)}(x_1-\bar{x}_1) + \frac{\partial f_i}{\partial x_2}\bigg|_{(\bar{x}_1, \bar{x}_2)}(x_2-\bar{x}_2) \\
  \end{align*}
  \begin{empheq}[box=\fbox]{align}
    \nonumber f_{1, lin}(x_1,x_2) &= (x_1 -\bar{x}_1) - 2x_2\bigg|_{(\bar{x}_1,\bar{x}_2)}(x_2 -\bar{x}_2) \\
    \nonumber f_{2, lin}(x_1,x_2) &= 2x_1\bigg|_{(\bar{x}_1,\bar{x}_2)}(x_1 -\bar{x}_1) + (6-14x_2)\bigg|_{(\bar{x}_1,\bar{x}_2)}(x_2 -\bar{x}_2)
  \end{empheq} 

  %   \begin{bmatrix}
  %     \dot{x_1} \\
  %     \dot{x_2}
  %   \end{bmatrix} &=
  %   \begin{bmatrix}
  %     x_1 & -2x_2 \\
  %     2x_1 & 6-14x_2
  %   \end{bmatrix}\bigg|_{(x_1,x_2)}
  %   \begin{bmatrix}
  %     x_1 \\
  %     x_2
  %   \end{bmatrix} 
    \newpage
 
  \textbf{\underline{Equilibrium Point: (9,\hspace{0.8mm}-3)}}

  \begin{align*}
    f_{1, lin}(x_1,x_2) &= (x_1 - 9) + 6(x_2 + 3) \\
    f_{2, lin}(x_1,x_2) &= 18(x_1 - 9) + 48(x_2 + 3)
  \end{align*}

  \indent Now we will define $\tilde{x}_1, \tilde{x}_2$, s.t.

  \begin{align*}
    \tilde{x}_1 &= x_1-9 \implies \dot{\tilde{x}}_1 = \dot{x}_1 \\
    \tilde{x}_2 &= x_2+3 \implies \dot{\tilde{x}}_2 = \dot{x}_2 \\
  \end{align*}
  
  \indent Plugging in gives us
  \begin{align*}
    \begin{bmatrix}
      \dot{\tilde{x}}_1 \\
      \dot{\tilde{x}}_2
    \end{bmatrix} &=
    \begin{bmatrix}
      \tilde{x}_1 + 6\tilde{x}_2 \\
      18\tilde{x}_1 + 48\tilde{x}_2
    \end{bmatrix} \\
    \begin{bmatrix}
      \dot{\tilde{x}}_1 \\
      \dot{\tilde{x}}_2
    \end{bmatrix} &=
    \begin{bmatrix}
      1 & 6 \\
      18 & 48
    \end{bmatrix} 
    \begin{bmatrix}
      \tilde{x}_1 \\
      \tilde{x}_2
    \end{bmatrix} 
  \end{align*}


  The eigenvalues of the matrix $\begin{bmatrix}
      1 & 6 \\
      18 & 48
  \end{bmatrix}$ are $\lambda_{1,2} = -1.20, 50.82$ \newline \newline
  \hangindent=1.58em
  \hangafter=1 Because one of the eignevalues has negative real part and one of
  the eigenvalues has positive real part, we know that \textbf{this
  equilibrium point is a saddle point.} \newline

The phase portrait near the point looks like the following: \hspace{4mm}
\includegraphics[height=45mm]{HW2_2-9-3.png} \newline \newline

  \textbf{\underline{Equilibrium Point: (0,\hspace{0.8mm}0)}}

  \begin{align*}
    f_{1, lin}(x_1,x_2) &= x_1 \\
    f_{2, lin}(x_1,x_2) &= 6x_2
  \end{align*}

  \begin{align*}
    \begin{bmatrix}
      \dot{\tilde{x}}_1 \\
      \dot{\tilde{x}}_2
    \end{bmatrix} &=
    \begin{bmatrix}
      x_1 \\
      6x_2
    \end{bmatrix} \\
    \begin{bmatrix}
      \dot{\tilde{x}}_1 \\
      \dot{\tilde{x}}_2
    \end{bmatrix} &=
    \begin{bmatrix}
      1 & 0 \\
      0 & 6
    \end{bmatrix} 
    \begin{bmatrix}
      \tilde{x}_1 \\
      \tilde{x}_2
    \end{bmatrix} 
  \end{align*}


  The eigenvalues of the matrix $\begin{bmatrix}
      1 & 0 \\
      0 & 6
  \end{bmatrix}$ are $\lambda_{1,2} = 1, 6$ \newline \newline
  \hangindent=1.58em
  \hangafter=1 Because both of the eignevalues have positive real parts  we know that \textbf{this
  equilibrium point is an unstable node.} \newline
  
  The phase portrait near the point looks like the following: \hspace{4mm}
  \includegraphics[height=45mm]{HW2_2-00.png} \newline \newline

 \textbf{\underline{Equilibrium Point: (1,\hspace{0.8mm}1)}}

  % \begin{align*}
  %   \begin{bmatrix}
  %     \dot{x_1} \\
  %     \dot{x_2}
  %   \end{bmatrix} &=
  %   \begin{bmatrix}
  %     1 & -2 \\
  %     2 & -8
  %   \end{bmatrix}\bigg|_{(x_1,x_2)}
  %   \begin{bmatrix}
  %     x_1 \\
  %     x_2
  %   \end{bmatrix} 
  %     \end{align*}

  \begin{align*}
    f_{1, lin}(x_1,x_2) &= (x_1 -_1) - 2(x_2 - 1) \\
    f_{2, lin}(x_1,x_2) &= 2(x_1 - 1) - 8(x_2 - 1)
  \end{align*}

  \indent Now we will define $\tilde{x}_1, \tilde{x}_2$, s.t.

  \begin{align*}
    \tilde{x}_1 &= x_1-1 \implies \dot{\tilde{x}}_1 = \dot{x}_1 \\
    \tilde{x}_2 &= x_2-1 \implies \dot{\tilde{x}}_2 = \dot{x}_2 \\
  \end{align*}
  
  \indent Plugging in gives us
  \begin{align*}
    \begin{bmatrix}
      \dot{\tilde{x}}_1 \\
      \dot{\tilde{x}}_2
    \end{bmatrix} &=
    \begin{bmatrix}
      \tilde{x}_1 - 2\tilde{x}_2 \\
      2\tilde{x}_1 - 8\tilde{x}_2
    \end{bmatrix} \\
    \begin{bmatrix}
      \dot{\tilde{x}}_1 \\
      \dot{\tilde{x}}_2
    \end{bmatrix} &=
    \begin{bmatrix}
      1 & -2 \\
      2 & -8
    \end{bmatrix} 
    \begin{bmatrix}
      \tilde{x}_1 \\
      \tilde{x}_2
    \end{bmatrix} 
  \end{align*}


  The eigenvalues of the matrix $\begin{bmatrix}
      1 & -2 \\
      2 & -8
  \end{bmatrix}$ are $\lambda_{1,2} = -7.53, 0.53$ \newline \newline
  \hangindent=1.58em
  \hangafter=1 Because one of the eignevalues has negative real part and one of
  the eigenvalues has positive real part, we know that \textbf{this
  equilibrium point is a saddle point.} \newline \newline
  
  The phase portrait near the point looks like the following: \hspace{4mm}
  \includegraphics[height=45mm]{HW2_2-11.png} \newpage

  \textbf{\underline{Equilibrium Point: (4,\hspace{0.4mm} 2)}}

  \begin{align*}
    f_{1, lin}(x_1,x_2) &= (x_1 - 4) - 4(x_2 - 2) \\
    f_{2, lin}(x_1,x_2) &= 8(x_1 - 4) - 22(x_2 - 2)
  \end{align*}

  \indent Now we will define $\tilde{x}_1, \tilde{x}_2$, s.t.

  \begin{align*}
    \tilde{x}_1 &= x_1-2 \implies \dot{\tilde{x}}_1 = \dot{x}_1 \\
    \tilde{x}_2 &= x_2-4 \implies \dot{\tilde{x}}_2 = \dot{x}_2 \\
  \end{align*}
  
  \indent Plugging in gives us
  \begin{align*}
    \begin{bmatrix}
      \dot{\tilde{x}}_1 \\
      \dot{\tilde{x}}_2
    \end{bmatrix} &=
    \begin{bmatrix}
      \tilde{x}_1 - 4\tilde{x}_2 \\
      8\tilde{x}_1 - 22\tilde{x}_2
    \end{bmatrix} \\
    \begin{bmatrix}
      \dot{\tilde{x}}_1 \\
      \dot{\tilde{x}}_2
    \end{bmatrix} &=
    \begin{bmatrix}
      1 & -4 \\
      8 & -22 
    \end{bmatrix} 
    \begin{bmatrix}
      \tilde{x}_1 \\
      \tilde{x}_2
    \end{bmatrix} 
  \end{align*}

  % \newgeometry{bottom=10mm}
  The eigenvalues of the matrix $\begin{bmatrix}
      1 & -4 \\
      8 & -22 
  \end{bmatrix}$ are $\lambda_{1,2} = -20.51, -0.49$
  \hangindent=1.58em
  \hangafter=1 Because both of the eignevalues has negative real parts, we know that \textbf{this
  equilibrium point is a stable node.} \newline
 
  The phase portrait near the point looks like the following: \hspace{4mm}
  \includegraphics[height=45mm]{HW2_2-42.png} \newpage

  % \restoregeometry

  \section{Problem 3}

  Consider the following mechanical system with position dependent damping and
  stiffness
  \begin{align*}
    \ddot{q} + c(q)\dot{q} + k(q) = 0 \tag{3.1}
  \end{align*}

  \noindent Show that if $c(q) > 0$ for all q, the system has no limit cycles. \newline

  \noindent \textit{Solution} \newline \newline
  \indent First, let's put the system in the state-space form. Let $x_1=q$,
  $x_2=\dot{q}$, then
  \begin{align*}
    \dot{x_1} &= f_1(x_1,x_2) = x_2 \tag{3.2} \\
    \dot{x_2} &= f_2(x_1,x_2) =  -c(q)x_2 - k(x_1) \tag{3.2} 
  \end{align*}

  \textbf{Bendixson's Criteria} states that if
  $(\frac{\partial{f_1}}{\partial{x_1}} + \frac{\partial{f_2}}{\partial{x_2}})$
  does not change sign in a region R, then no limit cycles exist.
  
  \begin{align*}
    \frac{\partial{f_1}}{\partial{x_1}} + \frac{\partial{f_2}}{\partial{x_2}} = -c(q) \tag{3.3}
  \end{align*}

  Because we know that $c(q) > 0$ for all q, we know that Bendixson's Criteria is
  satisfied for all \newline $(x_1, x_2) \in \mathbb{R}$. Therefore, there are
  no limit cycles for this system.

  \newpage

  \section{Problem 4}

  For the system below:
  \begin{align*}
    \dot{x_1} = 4x_1x_2^2 \tag{4.1} \\
    \dot{x_2} = 4x_1^2x_2 \tag{4.2} 
  \end{align*}
  
  \begin{enumerate}[label=(\alph*)]
    \item Find all the equilibrium points. Are they isolated?
    \item Show that the system has no limit cycles
  \end{enumerate}

  \noindent \textit{Solution}
  
  a) The equilibrium points are found by setting $\dot{x_1}=\dot{x_2}=0$.

  \begin{align*}
    \dot{x_1} &= 4x_1x_2^2 = 0 \implies x_1=0 \text{ or } x_2=0 \tag{4.3} \\
    \dot{x_2} &= 4x_1^2x_2 = 0 \implies x_1=0 \text{ or } x_2=0 \tag{4.4} \\
  \end{align*}

  \begin{empheq}[box=\fbox]{align}
    \nonumber \text{Equilibrium Points}: (0,a), (b,0)
  \end{empheq} \newline

  where $a,b \in \mathbb{R}$ \newline

  b) As stated in problem 3, Bendixson's Criteria states that if
  $(\frac{\partial{f_1}}{\partial{x_1}} + \frac{\partial{f_2}}{\partial{x_2}})$
  does not change sign in a region R, then no limit cycles exist.
  
  \begin{align*}
    \frac{\partial{f_1}}{\partial{x_1}} + \frac{\partial{f_2}}{\partial{x_2}} = 4x_2^2 + 4x_1^2 > 0 \tag{4.5} 
  \end{align*}

  Because $x_1^2$ and $x_2^2$ are both always greater than zero, we know that Bendixson's Criteria is
  satisfied for all $(x_1, x_2) \in \mathbb{R}$. Therefore, there are
  no limit cycles for this system.

  \newpage

  \section{Problem 5}

  Consider the nonlinear system

  \begin{align*}
    \dot{x_1} &= x_2 \tag{5.1} \\
    \dot{x_2} &= ax_1 + bx_2 - x_1^2x_2 - x_1^3 \tag{5.2} 
  \end{align*}

  \noindent where $a, b$ are constants. Find condition on a and b s.t. the system has no
  limit cycles in the phase plane. \newline \newline

  \noindent \textit{Solution} \newline

  As stated in problem 3, Bendixson's Criteria states that if
  $(\frac{\partial{f_1}}{\partial{x_1}} + \frac{\partial{f_2}}{\partial{x_2}})$
  does not change sign in a region R, then no limit cycles exist.
  
  \begin{align*}
    \frac{\partial{f_1}}{\partial{x_1}} + \frac{\partial{f_2}}{\partial{x_2}} = 1 + b -x_1^2 \tag{5.3} 
  \end{align*}

  We can see that Bendixson's Criteria does not depend on the value for a and
  that for the system to have no limit cycles in the phase plane b must be
  defined to be less than -1.

  \begin{empheq}[box=\fbox]{align}
    \nonumber a \in \mathbb{R}, b < -1
  \end{empheq} \newline

  


\end{document}

