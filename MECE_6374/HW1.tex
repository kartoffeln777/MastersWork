% This is a Latex tutorial document

\documentclass{article}
% \usepackage[letterpaper, landscape, margin=2in]{geometry}
\usepackage{amsmath}
\usepackage{arydshln}
\usepackage{mathtools}
\usepackage{graphicx}
\usepackage{pgfplots}
\usepackage{tikz}
\graphicspath{ {/Users/ericeldridge/Documents/MATLAB/MECE6374} }
\title{MECE 6374: Fun Work \#1}
\date{\today}
\author{Eric Eldridge (1561585)}
\begin{document}
  \maketitle

  \section{}

  Write the complete response of the following unforced scalar linear system. Confirm that it represents a solution. 

  \begin{equation*}
    \dot{x} = -4x, \hspace{5mm}x(0)=2
  \end{equation*}

  \noindent \textit{Solution}

  \begin{align}
    \frac{\dot{x}}{x} &= -4 \tag{1.1} \\
    \int \frac{\dot{x}}{x}dt &= \int -4dt \tag{1.2} \\
    ln(x) &= -4t+C \tag{1.3} \\
    x &= e^{-4t+C} = e^{-4t}e^C \tag{1.4}\\
    x &= Ce^{-4t} \tag{1.5} \\
    x(0) &= Ce^{-4*0} = 2 \tag{1.6} \\
    C &= 2 \tag{1.7} \\
    x(t) &= 2e^{-4t} \tag{1.8}
  \end{align}

  To confirm that this represents a solution, let's plug x(t) into our system and confirm that the system holds.

  \begin{align}
    \frac{\dot{x}}{x} &= -4 \tag{1.1} \\
    x(t) &= 2e^{-4t} \tag{1.8} \\
    \dot{x}(t) &= -8e^{-4t} \tag{1.9}
  \end{align}

  Plugging into

  \begin{equation}
    \frac{-8e^{-4t}}{2e^{-4t}} = -4 \tag{1.10}
  \end{equation}
   
  and we can see that the equation holds. 
  
  \newpage
  \section{}

  Write a state-space representation of the following dynamic systems:
  \newline
  \newline
  a) $4\ddot{q}+5\dot{q}+8q=0$
  \newline
  \newline
  Let $q_1 = q$ and $q_2 = \dot{q}$
  We can plug $q_1$ and $q_2$ into equation (a) to get
  \begin{align}
    4\dot{q_2} &+ 5q_2 + 8q_1 = 0 \tag{2.1}\\
    4\dot{q_2} &= -5q_2 -8q_1 \tag{2.2} \\
    \dot{q_2}  &= \frac{-5}{4}q_2 - 2q_1 \tag{2.3} 
  \end{align}
  
  \noindent We solved for $q_2$ above and we know $\dot{q_1} = q_2$
  % \dot{q_1} = q_2 \\
  \newline
  Let $q = \begin{bmatrix}
    q_1 \\
    q_2
  \end{bmatrix}$. Then

  \begin{align}
    \dot{q_1} &= q_2 \tag{2.4} \\
    \dot{q_2} &= -\frac{5}{4}q_2 - 2q_1 \tag{2.5}
  \end{align}
  \begin{align}
    \begin{bmatrix}
      \dot{q_1} \\
      \dot{q_2}
    \end{bmatrix} =
    \begin{bmatrix}
      0 & 1 \\
      -2 & -\frac{5}{4}
    \end{bmatrix}*
    \begin{bmatrix}
      q_1 \\
      q_2 \tag{2.6}
    \end{bmatrix}
  \end{align}
  .
  \newline \newline
  \noindent b) $4\ddot{q}+5\dot{q}+8q^3 = 0$
  \newline \newline
  \indent Let $q_1 = q$ and $q_2 = \dot{q}$
  We can plug $q_1$ and $q_2$ into equation (b) to get
  \begin{align*}
    4\dot{q_2} &+ 5q_2 + 8q_1^3 = 0 \tag{2.7} \\
    4\dot{q_2} &= -5q_2 -8q_1^3 \tag{2.8} \\
    \dot{q_2}  &= -\frac{5}{4}q_2 - 2q_1^3 \tag{2.9} 
  \end{align*}
  \begin{align*}
    \dot{q_1} &= q_2 \tag{2.10} \\
    \dot{q_2} &= -\frac{5}{4}q_2 - 2q_1^3 \tag{2.11}
  \end{align*}
     
  This can not be split up any further as in part a) because of the
  non-linearity of the system.
  
  \newpage 
  \noindent c) $2\dddot{q}+4\ddot{q}+5\dot{q}+8q = 0$
  \newline \newline
  \indent Let $q_1 = q$ and $q_2 = \dot{q} and q_3 = \ddot{q}$
  We can plug $q_1$, $q_2$, and $q_3$ into equation (c) to get
  \begin{align*}
    2\dot{q_3} &+ 4q_3 + 5q_2 + 8q_1 = 0 \tag{2.12}\\
    2\dot{q_3} &= -4q_3 -5q_2 - 8q_1 \tag{2.13} \\
    \dot{q_3}  &= -2q_3 -\frac{5}{2}q_2 - 4q_1 \tag{2.14} \\
  \end{align*}
  \begin{align}
    \dot{q_1} &= q_2 \tag{2.15} \\
    \dot{q_2} &= q_3 \tag{2.16} \\
    \dot{q_3} &= -4q_1 -\frac{5}{2}q_2 - 2q_3 \tag{2.17}
  \end{align}
  \begin{align}
    \begin{bmatrix}
      \dot{q_1} \\
      \dot{q_2} \\
      \dot{q_3}
    \end{bmatrix} =
    \begin{bmatrix}
      0 & 1 & 0\\
      0 & 0 & 1 \\
      -4 & -\frac{5}{2} & -2
    \end{bmatrix}*
    \begin{bmatrix}
      q_1 \\
      q_2 \\
      q_3 \tag{2.18}
    \end{bmatrix}
  \end{align}

  \newpage
  \section{}
  Find the equilibrium points of the following systems. Show the equilibrium
  points on the phase plane.
  \newline \newline
  a)
  \begin{flalign*}
    \dot{x_1} = -2x_1 +x_1x_2 \\
    \dot{x_2} = x_2 -2x_1x_2
  \end{flalign*}

  The equilibrium point is found by setting $\dot{x_1} = \dot{x_2} = 0$
  \begin{align*}
    \dot{x_1} &= 0 \\
    -2x_1 + x_1x_2 &= 0 \\
    x_1(-2+x_2)   &= 0 \\
    x_1 = 0 \hspace{2mm} or \hspace{2mm} x_2 &= 2 \tag{3.1} \\~\\
    \dot{x_2} &= 0 \\
    x_2 -2x_1x_2 &= 0 \\
    x_2(1-2x_1) &= 0 \\
    x_2 = 0 \hspace{2mm} or \hspace{2mm} x_1 &= \frac{1}{2} \tag{3.2}
  \end{align*}

  The only intersection between these two sets of solutions is the point (0,0)
  \newline \newline 


  b)
  \begin{flalign*}
    \dot{x_1} &= x_1 -x_2 \\
    \dot{x_2} &= 1-x_1x_2
  \end{flalign*}
  
  The equilibrium point is found by setting $\dot{x_1} = \dot{x_2} = 0$
  \begin{align*}
    \dot{x_1} &= 0 \\
    x_1 - x_2 &= 0 \\
    x_1 &= x_2 \tag{3.3} \\~\\
    \dot{x_2} &= 0 \\
    1 - x_1x_2 &= 0 \\
    x_1x_2 &= 1 \tag{3.4} \\
  \end{align*}

  \newpage Plugging in (3.3) into (3.4)

  \begin{align*}
    x_1^2 &= 1 \\
    x_1 &= \pm 1 \\
    x_1 = x_2 \rightarrow x_2 &= \pm 1 \tag{3.5}
  \end{align*}

  There are two equilibrium points: (-1,\hspace{1mm}-1) and (1,\hspace{1mm}1)
  \newline \newline

  \begin{center}
    \begin{tikzpicture}
      \begin{axis}[
          axis lines=middle,
          xmin=-3, xmax=3,
          ymin=-3, ymax=3,
          xtick={-2,-1,0,1,2}, ytick={-2,-1,0,1,2}
      ]
     \addplot [only marks] table {
      -1  -1
       1   1
      };
      \end{axis}
    \end{tikzpicture}  
  \end{center}

  \newpage \noindent c)
  \begin{flalign*}
    \dot{x_1} = x_2(x_1^2+x_2^2-1) \\
    \dot{x_2} = x_1(x_1^2+x_2^2-1)
  \end{flalign*}

  The equilibrium point is found by setting $\dot{x_1} = \dot{x_2} = 0$
  \begin{align*}
    \dot{x_1} &= 0 \\
    x_2 = 0 \hspace{2mm} or \hspace{2mm} x_1^2 + x_2^2 &= 1 \tag{3.6} \\~\\
    \dot{x_2} &= 0 \\
    x_1 = 0 \hspace{2mm} or \hspace{2mm} x_1^2 + x_2^2 &= 1 \tag{3.7}
  \end{align*}

  The equilibrium points lie at the point (0, \hspace{1mm} 0) and on the unit
  circle. \newline \newline

  \begin{center}
    \begin{tikzpicture}
      \begin{axis}[
          axis lines=middle,
          xmin=-3, xmax=3,
          ymin=-3, ymax=3,
          xtick={-2,-1,0,1,2}, ytick={-2,-1,0,1,2}
      ]
      \addplot [only marks] table {
        0   0
      };
      %\addplot [domain=-0.9:0.9] {(1-x^2)^{0.5}};
      \addplot [domain=0:2*pi,samples=50]({cos(deg(x))},{sin(deg(x))});
      \end{axis}
    \end{tikzpicture}  
  \end{center}

  \newpage
  \section{}

  Consider the Van der Pol equation:
  \begin{align*}
    \ddot{y} - (1-y^2)\dot{y} + y = 0
  \end{align*}

  where $y=y(t)$. A state space representation of this equation is
  \begin{align*}
    \dot{x_1} &= x_2 \\
    \dot{x_2} &= -x_1 + (1-x_1^2)x_2
  \end{align*}

  where $x_1=y$ and $x_2=\dot{y}$. Integrate numerically the state space
  equations with initial conditions
  \begin{align*}
    (x_1(0), x_2(0)) = (0.1,0.1), (1,1), (-1,1), (1,-1), (-1,-1), (3,3), (-3,3), (3,-3), (-3,-3)
  \end{align*}
  respectively to obtain a representation of the phase plane portrait.

  %\insertgraphics{vdp1}
  
  
\end{document}

