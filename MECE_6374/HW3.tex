% This is a Latex tutorial document

\documentclass{article}
% \usepackage[letterpaper, landscape, margin=2in]{geometry}
\usepackage[margin=1.0in]{geometry}
\usepackage{amsmath}
\usepackage{amssymb}
\usepackage{empheq}
\usepackage{mathtools}
\usepackage{graphicx}
\usepackage{pgfplots}
\usepackage{cancel}
\usepackage{enumitem}
\graphicspath{ {./../../../Documents/GradSchool/MECE6374/} }
\makeatletter
\def\@seccntformat#1{%
  \expandafter\ifx\csname c@#1\endcsname\c@section\else
  \csname the#1\endcsname\quad
  \fi}
\makeatother
\title{MECE 6374: Fun Work \#3}
\date{\today}
\author{Eric Eldridge (1561585)}
\begin{document}

  % \setlength{\abovedisplayshortskip}{3pt}
  % \setlength{\belowdisplayshortskip}{3pt}
  % \setlength{\abovedisplayskip}{3pt}
  % \setlength{\belowdisplayskip}{3pt}

  \maketitle

  \section{Problem 1}

  Consider the rotational equations of a rigid body about its principal axes.

  \begin{align*}
    \dot{\omega_1} &= 10\omega_2\omega_3 \\
    \dot{\omega_2} &= -5\omega_3\omega_1 \\
    \dot{\omega_3} &= 2\omega_1\omega_2
  \end{align*}

  \begin{enumerate}[label=(\alph*)]
    \item Find the equilibrium points of this system.
    \item Linearize the system about these equilibrium points. Can you determine
      the stability of the these equilibrium points from the linearized systems?
  \end{enumerate} 
    
  \vspace{3mm}
  \noindent \textit{Solution} \newline 

  \indent (a) The equilibrium points $(\bar{\omega}_1, \bar{\omega}_2,
  \bar{\omega}_3)$ occur when
  $\dot{\omega} =
  \begin{bmatrix}
      \dot{\omega_1} \\
      \dot{\omega_2} \\
      \dot{\omega_3}
  \end{bmatrix} =
  \begin{bmatrix}
      0 \\
      0 \\
      0
  \end{bmatrix} $

  \begin{align*}
    10\omega_2\omega_3 &= 0 \\
    \implies \omega_2 &= 0, \text{ or } \omega_3 = 0 \\
    -5\omega_3\omega_1 &= 0 \\
    \implies \omega_1 &= 0, \text{ or } \omega_3 = 0 \\
    2\omega_1\omega_2 &= 0 \\
    \implies \omega_1 &= 0, \text{ or } \omega_2 = 0 \\
  \end{align*}

  This tells us that so long as any two of $\omega_1$, $\omega_2$, $\omega_3$
  are set to 0, there is an equilibrium point.

  \begin{empheq}[box=\fbox]{align}
    \nonumber \text{Equilibrium Points}: (\alpha ,0, 0), (0, \beta, 0), (0,0,
    \gamma) \\
    \nonumber \text{where $\alpha$, $\beta$, $\gamma$} \in \mathbb{R}
  \end{empheq} \newpage

  (b) If we define functions $f_1, f_2, f_3$, s.t.
  $\dot{\omega} =
  \begin{bmatrix}
      \dot{\omega_1} \\
      \dot{\omega_2} \\
      \dot{\omega_3}
  \end{bmatrix} =
  \begin{bmatrix}
      f_1(\omega_1, \omega_2, \omega_3) \\
      f_2(\omega_1, \omega_2, \omega_3) \\
      f_3(\omega_1, \omega_2, \omega_3)
  \end{bmatrix} $

  The formula for linearization of $\dot{\omega}$ is
  \begin{align*}
    f_{i, lin}(\omega_1,\omega_2, \omega_3) = \cancelto{0}{f_i(\bar{\omega}_1, \bar{\omega}_2, \bar{\omega}_3)} \hspace{3.0mm} + \hspace{1.0mm}
    \frac{\partial f_i}{\partial \omega_1}\bigg|_{(\bar{\omega}_1, \bar{\omega}_2, \bar{\omega}_3)}(\omega_1-\bar{\omega}_1) +
    \frac{\partial f_i}{\partial \omega_2}\bigg|_{(\bar{\omega}_1, \bar{\omega}_2, \bar{\omega}_3)}(\omega_2-\bar{\omega}_2) \hspace{1.0mm} + \cdots \\
    \cdots \hspace{1.0mm} + \hspace{1.0mm} \frac{\partial f_i}{\partial \omega_3}\bigg|_{(\bar{\omega}_1, \bar{\omega}_2, \bar{\omega}_3)}(\omega_3-\bar{\omega}_3) \hspace{1.0mm} + \hspace{1.0mm}
    \cancelto{0}{H.O.T.} \\ %\tag{1.8} \\
    % \text{\vspace{3.0mm}}
    f_{i, lin}(\omega_1,\omega_2, \omega_3) =
    \frac{\partial f_i}{\partial \omega_1}\bigg|_{(\bar{\omega}_1, \bar{\omega}_2, \bar{\omega}_3)}(\omega_1-\bar{\omega}_1) +
    \frac{\partial f_i}{\partial \omega_2}\bigg|_{(\bar{\omega}_1, \bar{\omega}_2, \bar{\omega}_3)}(\omega_2-\bar{\omega}_2) \hspace{1.0mm} + \hspace{1.0mm}
    \frac{\partial f_i}{\partial \omega_3}\bigg|_{(\bar{\omega}_1, \bar{\omega}_2, \bar{\omega}_3)}(\omega_3-\bar{\omega}_3) \\
  \end{align*}
  \begin{align*}
    f_{1, lin}(\omega_1,\omega_2, \omega_3) &=
    \hspace{1.4mm} 10\omega_3\bigg|_{(\bar{\omega}_1, \bar{\omega}_2, \bar{\omega}_3)}(\omega_2-\bar{\omega}_2) \hspace{1.0mm} + 
    10\omega_2\bigg|_{(\bar{\omega}_1, \bar{\omega}_2, \bar{\omega}_3)}(\omega_3-\bar{\omega}_3) \\
    f_{2, lin}(\omega_1,\omega_2, \omega_3) &=
    -5\omega_3\bigg|_{(\bar{\omega}_1, \bar{\omega}_2, \bar{\omega}_3)}(\omega_1-\bar{\omega}_1) \hspace{1.0mm} - \hspace{2.0mm}
    5\omega_1\bigg|_{(\bar{\omega}_1, \bar{\omega}_2, \bar{\omega}_3)}(\omega_3-\bar{\omega}_3) \\
    f_{3, lin}(\omega_1,\omega_2, \omega_3) &=
    \hspace{2.5mm} 2\omega_2\bigg|_{(\bar{\omega}_1, \bar{\omega}_2, \bar{\omega}_3)}(\omega_1-\bar{\omega}_1) \hspace{1.0mm} + \hspace{1.0mm}
    \hspace{1.5mm} 2\omega_1\bigg|_{(\bar{\omega}_1, \bar{\omega}_2, \bar{\omega}_3)}(\omega_2-\bar{\omega}_2) \\
     \end{align*}

  \textbf{\underline{Equilibrium Point: ($\alpha$,\hspace{0.8mm}0,\hspace{0.8mm}0)}}

  \begin{align*}
    f_{1, lin}(\omega_1,\omega_2, \omega_3) &= \dot{\omega}_{1,lin} = 0 \\
    f_{2, lin}(\omega_1,\omega_2, \omega_3) &= \dot{\omega}_{2,lin} =  -5\alpha\omega_3 \\
    f_{3, lin}(\omega_1,\omega_2, \omega_3) &= \dot{\omega}_{3,lin} = 2\alpha\omega_2 \\
    \begin{bmatrix}
      \dot{\omega}_{1,lin} \\
      \dot{\omega}_{2,lin} \\
      \dot{\omega}_{3,lin}
    \end{bmatrix} &=
    \begin{bmatrix}
      0 & 0 & 0 \\
      0 & 0 & -5\alpha \\
      0 & 2\alpha & 0
    \end{bmatrix}
    \begin{bmatrix}
      \omega_1 \\
      \omega_2 \\
      \omega_3
    \end{bmatrix}
  \end{align*}
    
  \noindent The eigenvalues of the matrix 
  $\begin{bmatrix}
    0 & 0 & 0 \\
    0 & 0 & -5\alpha \\
    0 & 2\alpha & 0
  \end{bmatrix}$
  are $\lambda_{1,2,3} = 0, \pm \sqrt{10}\alpha j$ \vspace{3.0mm}
  These eigenvalues tell us that the linearized system is marginally stable
  which does not tell us anything about the behavior of the non-linear system
  near the equilibrium point $(\alpha, 0, 0)$. \newpage
    
  \textbf{\underline{Equilibrium Point: (0,\hspace{0.8mm}$\beta$,\hspace{0.8mm}0)}}

  \begin{align*}
    f_{1, lin}(\omega_1,\omega_2, \omega_3) &= \dot{\omega}_{1,lin} = 10\beta\omega_3 \\
    f_{2, lin}(\omega_1,\omega_2, \omega_3) &= 0 \\
    f_{3, lin}(\omega_1,\omega_2, \omega_3) &= \dot{\omega}_{3,lin} = 2\beta\omega_1 \\
    \begin{bmatrix}
      \dot{\omega}_{1,lin} \\
      \dot{\omega}_{2,lin} \\
      \dot{\omega}_{3,lin}
    \end{bmatrix} &=
    \begin{bmatrix}
      0 & 0 & 10\beta \\
      0 & 0 & 0 \\
      2\beta & 0 & 0
    \end{bmatrix}
    \begin{bmatrix}
      \omega_1 \\
      \omega_2 \\
      \omega_3
    \end{bmatrix}
  \end{align*}
    
  \noindent The eigenvalues of the matrix 
  $\begin{bmatrix}
      0 & 0 & 10\beta \\
      0 & 0 & 0 \\
      2\beta & 0 & 0
  \end{bmatrix}$
  are $\lambda_{1,2,3} = 0, \pm \sqrt{20}\beta$ \vspace{3.0mm}
  These eigenvalues tell us that, no matter the value of $\beta$, the linearized system is unstable
  which tells us that the non-linear system near the equilibrium point $(0,
  \beta, 0)$ is unstable. \newline \newline
    

  \textbf{\underline{Equilibrium Point: (0,\hspace{0.8mm}0,\hspace{0.8mm}$\gamma$)}}

  \begin{align*}
    f_{1, lin}(\omega_1,\omega_2, \omega_3) &= \dot{\omega}_{1,lin} = 10\gamma\omega_2 \\
    f_{2, lin}(\omega_1,\omega_2, \omega_3) &= \dot{\omega}_{3,lin} = -5\gamma\omega_1 \\
    f_{3, lin}(\omega_1,\omega_2, \omega_3) &= 0 \\
    \begin{bmatrix}
      \dot{\omega}_{1,lin} \\
      \dot{\omega}_{2,lin} \\
      \dot{\omega}_{3,lin}
    \end{bmatrix} &=
    \begin{bmatrix}
      0 & 10\gamma & 0 \\
      -5\gamma & 0 & 0 \\
      0 & 0 & 0
    \end{bmatrix}
    \begin{bmatrix}
      \omega_1 \\
      \omega_2 \\
      \omega_3
    \end{bmatrix}
  \end{align*}
    
  \noindent The eigenvalues of the matrix 
  $\begin{bmatrix}
      0 & 10\gamma & 0 \\
      -5\gamma & 0 & 0 \\
      0 & 0 & 0
  \end{bmatrix}$
  are $\lambda_{1,2,3} = 0, \pm 5\sqrt{2}\gamma$ \vspace{3.0mm}
  These eigenvalues tell us that the linearized system is marginally stable
  which does not tell us anything about the behavior of the non-linear system
  near the equilibrium point $(\alpha, 0, 0)$. \newpage
    

  \section{Problem 2}

  Consider the following Loventz attractor system.

  \begin{align*}
    \dot{x}_1 &= -\sigma x_1 + \sigma x_2 \\
    \dot{x}_2 &= \rho x_1 - x_2 - x_1x_3 \\
    \dot{x}_3 &= -\beta x_3 + x_1x_2
  \end{align*}

  where $\sigma = 10$, $\beta = \frac{8}{3}$, and $\rho$ is a parameter. Compute
  the equilibrium points of the system. How do these equilibrium points change
  as $\rho$ varies from 0 to $\infty$?

  \vspace{3mm}
  \noindent \textit{Solution} \newline 

  \indent The equilibrium points $(\bar{x}_1, \bar{x}_2,
  \bar{x}_3)$ occur when
  $\dot{x} =
  \begin{bmatrix}
      \dot{x_1} \\
      \dot{x_2} \\
      \dot{x_3}
  \end{bmatrix} =
  \begin{bmatrix}
      0 \\
      0 \\
      0
  \end{bmatrix} $

  \begin{align*}
    -\sigma \bar{x}_1 + \sigma \bar{x}_2 &= 0 \\
    \implies \bar{x}_1 &= \bar{x}_2 \\
    \rho \bar{x}_1 - \bar{x}_2 - \bar{x}_1\bar{x}_3 &= 0 \\
    \rho \bar{x}_2 - \bar{x}_2 - \bar{x}_2\bar{x}_3 &= 0 \\
    \bar{x}_2(\rho - 1 - \bar{x}_3) &= 0 \\
    \implies \bar{x}_2 &= 0, \text{ or } \bar{x}_3 = \rho - 1 \\
    -\beta \bar{x}_3 + \bar{x}_1\bar{x}_2 &= 0 \\
    -\beta \bar{x}_3 + \bar{x}_2^2 &= 0 \\
    \bar{x}_2 = 0 \implies \bar{x}_1,\bar{x}_3 &= 0, \\
    \bar{x}_3 = \rho - 1 \implies -\beta (\rho - 1) + \bar{x}_2^2 &= 0 \implies \bar{x}_2 = \bar{x}_1 = \pm \sqrt{\beta (\rho-1)}
  \end{align*}

  \begin{empheq}[box=\fbox]{align}
    \nonumber \text{Equilibrium Points}: (0,0,0), (\sqrt{\beta (\rho-1)},
    \sqrt{\beta (\rho-1)}, \rho-1), \\
    \nonumber (-\sqrt{\beta (\rho-1)}, -\sqrt{\beta (\rho-1)}, \rho-1)
  \end{empheq} \newline \newline

  \noindent For $\rho < 1$, the only equilibrium point is at the origin, $(0,0,0)$.
  \newline \newline
  For $\rho = 1$, there are equilibrium points at $(0,0,0), (0,0,-1)$ \newline \newline
  For $\rho > 1$, there are equilibrium points at $(0,0,0), (\sqrt{\beta (\rho-1)},
  \sqrt{\beta (\rho-1)}, \rho-1)$, \newline
  $(-\sqrt{\beta (\rho-1)}, -\sqrt{\beta (\rho-1)}, \rho-1)$





\end{document}

