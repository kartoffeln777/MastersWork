% This is a Latex tutorial document

\documentclass{article}
% \usepackage[letterpaper, landscape, margin=2in]{geometry}
\usepackage[margin=1.0in]{geometry}
\usepackage{amsmath}
\usepackage{amssymb}
\usepackage{empheq}
\usepackage{mathtools}
\usepackage{graphicx}
\usepackage{pgfplots}
\usepackage{cancel}
\usepackage{enumitem}
\graphicspath{ {./../../../Documents/GradSchool/MECE6374/HW4/} }
\makeatletter
\def\@seccntformat#1{%
  \expandafter\ifx\csname c@#1\endcsname\c@section\else
  \csname the#1\endcsname\quad
  \fi}
\makeatother
\title{MECE 6374: Fun Work \#4}
\date{\today}
\author{Eric Eldridge (1561585)}
\begin{document}

  % \setlength{\abovedisplayshortskip}{3pt}
  % \setlength{\belowdisplayshortskip}{3pt}
  % \setlength{\abovedisplayskip}{3pt}
  % \setlength{\belowdisplayskip}{3pt}

  \maketitle

  \section{Problem 1}

  Consider the following scalar nonlinear system

  \begin{align*}
    \dot{x} &= -x^3
  \end{align*}

  \noindent We are interested to examine the stability of the origin $\tilde{x}=0$.

  \begin{enumerate}[label=(\alph*)]
    \item Can you determine stability using the linearization of the nonlinear system?
    \item Consider the Lyapunov function
          \begin{align*}
            V(x) = x^4
          \end{align*}
          Use Lyapunov's Direct method to determine the stability (global or local)
          of the origin $\tilde{x}=0$.
  \end{enumerate} 

  \vspace{3mm}
  
  \noindent \textit{Solution} \newline \newline

  (a) If we let $\dot{x}=f(x)=-x^3$, the equilibrium point $\bar{x}$ occurs when
  $f(\bar{x})=0$ \newline
  \begin{align*}
    -x^3 = 0 \implies \bar{x}=0
  \end{align*}

  \indent The formula for linearization of $\dot{x}$ is

  \begin{align*}
    f_{lin}(x) &= \cancelto{0}{f(x)} \hspace{3.0mm} + \hspace{1.0mm}
    \frac{\partial f}{\partial x}\bigg|_{x=\bar{x}}(x-\bar{x}) + \cancelto{0}{H.O.T.} \\ 
    f_{lin}(x) &= \frac{\partial f}{\partial x}\bigg|_{x=\bar{x}}(x-\bar{x}) \\
     % \text{\vspace{3.0mm}}
    f_{lin}(x) &= -3x^2\bigg|_{x=0}x \\
    f_{lin}(x) &= 0
  \end{align*}

  The system $\dot{x} = 0x$ has the eigenvalue 0 which tells us that the
  linearized system is marginally stable which does not tell us anything about
  the behavior of the non-linear system near the equilibrium point 0. \newpage
    
  \noindent (b) Lyapunov's 2nd method (Energy Method) states that if $V(0)=0$  and $V(x)
  \geq 0$ for $x \neq 0$, then the function V(x) is said to have ``energy-like'' properties.
  This holds true for $V(x)=x^4$.

  Further, Lyapnuov's 2nd method states that if V(x) decreases and $V(x)=0$, then the system is stable.

  \begin{align*}
    V(x) &= x^4 \\
    \dot{V}(x) &= 4x^3\dot{x} \\
    \dot{V}(x) &= 4x^3(-x^3) = -4x^6
  \end{align*}

  We can see that $\dot{V}(x) < 0$ for all x. Therefore V(x) is global negative
  definite. V(x) is globally positive definite and radially unbound and
  $\dot{V}$ is globally negative definite which proves that \textbf{the system
    is globally asymptotically stable}. \newpage


  \section{Problem 2}

  Consider the nonlinear system
  \begin{align*}
    \dot{x}_1 = (x_1 - x_2)(x_1^2 + x_2^2 -1) \\
    \dot{x}_2 = (x_1 + x_2)(x_1^2 + x_2^2 -1)
  \end{align*}

  \begin{enumerate}[label=(\alph*)]
    \item Find all equilibrium points
    \item Use linearization and Lyapunov methods to show that (0, \hspace{0.5mm}0) is an
      asymptotically stable equilibrium.
    \item Is (0,0) globally stable?
  \end{enumerate}
    

  \vspace{3mm}
  
  \noindent \textit{Solution} \newline \newline

  (a) The equilibrium points $(\bar{x}_1, \bar{x}_2)$ occur when
  $\dot{x} =
  \begin{bmatrix}
      \dot{x}_1 \\
      \dot{x}_2 \\
  \end{bmatrix} =
  \begin{bmatrix}
      0 \\
      0
  \end{bmatrix} $

  \begin{align*}
    (x_1 - x_2)(x_1^2 + x_2^2 -1) = 0 \\
    \implies x_1 = x_2 \text{ or } x_1^2 + x_2^2 = 1 \\
    (x_1 + x_2)(x_1^2 + x_2^2 -1) = 0 \\
    x_1=x_2 \implies (x_2 + x_2)(x_2^2 + x_2^2 -1) = 0 \\
    (2x_2)(2x_2^2-1) = 0 \\
    \implies x_2=0 \text{ or } x_2 = \pm \frac{\sqrt{2}}{2} \\
    x_1^2 + x_2^2 = 1 \implies (x_1 - x_2)(0) = 0 \\
    \implies x_1, x_2 \text{ can be anything so long as } x_1^2 + x_2^2 = 1 \text{ is satisfied.}
  \end{align*}

  The points $(\frac{\sqrt{2}}{2}, \frac{\sqrt{2}}{2})$ and
  $(-\frac{\sqrt{2}}{2}, -\frac{\sqrt{2}}{2})$ both belong to the set of points
  of points described by $x_1^2 + x_2^2 = 1$. \newline
  \vspace{1mm}
  \indent Therefore, the equilibrium points are:
  
  \begin{empheq}[box=\fbox]{align}
    \nonumber \text{Equilibrium Points}: (0,0), (x_1, x_2) \text{ s.t. } x_1^2 + x_2^2 = 1 
  \end{empheq} \newline \newline

  \begin{center}
    \begin{tikzpicture}
      \begin{axis}[
          axis lines=middle,
          xmin=-3, xmax=3,
          ymin=-3, ymax=3,
          xtick={-2,-1,0,1,2}, ytick={-2,-1,0,1,2}
      ]
      \addplot [only marks] table {
        0   0
      };
      %\addplot [domain=-0.9:0.9] {(1-x^2)^{0.5}};
      \addplot [domain=0:2*pi,samples=50]({cos(deg(x))},{sin(deg(x))});
      \end{axis}
    \end{tikzpicture}  
  \end{center}

  \newpage

  (b) The formula for linearization of $\dot{x}$ is

  \begin{align*}
    f_{1, lin}(x) &= \cancelto{0}{f_1(\bar{x})} \hspace{3.0mm} + \hspace{1.0mm}
    \frac{\partial f_1}{\partial x_1}\bigg|_{x=\bar{x}}(x_1-\bar{x}_1) \hspace{3.0mm} + \hspace{1.0mm}
    \frac{\partial f_1}{\partial x_2}\bigg|_{x=\bar{x}}(x_2-\bar{x}_2)+ \cancelto{0}{H.O.T.} \\ 
    f_{2, lin}(x) &= \cancelto{0}{f_2(\bar{x})} \hspace{3.0mm} + \hspace{1.0mm}
    \frac{\partial f_2}{\partial x_1}\bigg|_{x=\bar{x}}(x_1-\bar{x}_1) \hspace{3.0mm} + \hspace{1.0mm}
    \frac{\partial f_2}{\partial x_2}\bigg|_{x=\bar{x}}(x_2-\bar{x}_2)+ \cancelto{0}{H.O.T.} \\
    f_{1, lin}(x) &= \frac{\partial f_1}{\partial x_1}\bigg|_{x=\bar{x}}(x_1-\bar{x}_1) \hspace{3.0mm} + \hspace{1.0mm}
    \frac{\partial f_1}{\partial x_2}\bigg|_{x=\bar{x}}(x_2-\bar{x}_2) \\ 
    f_{2, lin}(x) &= \frac{\partial f_2}{\partial x_1}\bigg|_{x=\bar{x}}(x_1-\bar{x}_1) \hspace{3.0mm} + \hspace{1.0mm}
    \frac{\partial f_2}{\partial x_2}\bigg|_{x=\bar{x}}(x_2-\bar{x}_1) \\     
     % \text{\vspace{3.0mm}}
    f_{1, lin}(x) &= ((x_1^2 + x_2^2 -1) + (x_1 - x_2)(2x_1))\bigg|_{(0,0)}x_1 + (-(x_1^2 + x_2^2 -1) + (x_1 - x_2)(2x_2))\bigg|_{(0,0)}x_2\\
    f_{2, lin}(x) &= ((x_1^2 + x_2^2 -1) + (x_1 - x_2)(2x_1))\bigg|_{(0,0)}x_1 + ((x_1^2 + x_2^2 -1) + (x_1 - x_2)(2x_2))\bigg|_{(0,0)}x_2\\
    f_{1, lin}(x) &= -x_1 + x_2 \\
    f_{2, lin}(x) &= -x_1 - x_2 \\
    \begin{bmatrix}
      \dot{x_1} \\
      \dot{x_2}
    \end{bmatrix} &=
    \begin{bmatrix}
      -x_1 + x_2 \\
      -x_1 - x_2
    \end{bmatrix} \\
    \begin{bmatrix}
      \dot{x_1} \\
      \dot{x_2}
    \end{bmatrix} &=
    \begin{bmatrix}
     -1 & 1 \\
     -1 & -1
    \end{bmatrix} 
    \begin{bmatrix}
      x_1 \\
      x_2
    \end{bmatrix} 
   \end{align*}

  The eigenvalues of the matrix $\begin{bmatrix}
      -1 & 1 \\
      -1&  -1
  \end{bmatrix}$ are $\lambda_{1,2} = -1 \pm 1j$
  \newline
  \hangindent=1.58em
  \hangafter=1 Because both of the eignevalues have negative real parts, we know that \textbf{this
  equilibrium point is asymptotically stable.} \newline \newline

  We can also use Lyapunov's 2nd Method to solve this. If we choose the energy function
  $V(x)=x_1^2+x_2^2$, we can see that V is global positive definite and radially
  unbounded. We can solve for $\dot{V}$:

  \begin{align*}
    \dot{V}(x) &= 2x_1\dot{x}_1 + 2x_2\dot{x}_2 \\
    \dot{V}(x) &= 2x_1(x_1 - x_2)(x_1^2 + x_2^2 -1) + 2x_2(x_1 + x_2)(x_1^2 + x_2^2 -1) \\
    \dot{V}(x) &= 2(x_1^2 + x_2^2)(x_1^2 +x_2^2 - 1)
  \end{align*}

  We can see that $\dot{V}$(x) is locally negative definite. Therefore (0,0) is
  \textbf{locally asymptotically stable}. \newline \newline

  (c) As shown using Lyapunov's 2nd method in the latter half of part (b), we
  can see that the equilibrium point is \underline{NOT} globally stable

  \newpage

  \section{Problem 3}

  Consider the following state-space system
  \begin{align*}
    \dot{x}_1 &= x_2 \\
    \dot{x}_2 &= -\frac{2x_1}{(1+x_1^2)^2}
  \end{align*}

  \begin{enumerate}[label=(\alph*)]
    \item Can you determine stability of the equilibrium point $\bar{x}=0$ using
      the Lyapunov function below?
      \begin{align*}
        V(x)=\frac{x_1^2}{1+x_1^2}+\frac{1}{2}x_2^2
      \end{align*}
    \item Can you determine global stability?
    \item Confirm the answers above by plotting the phase plane portrait of the
      system.
  \end{enumerate}

  \vspace{3mm}
  
  \noindent \textit{Solution} \newline \newline

  (a) The formula for linearization of $\dot{x}$ is

  \begin{align*}
    f_{1, lin}(x) &= \cancelto{0}{f_1(\bar{x})} \hspace{3.0mm} + \hspace{1.0mm}
    \frac{\partial f_1}{\partial x_1}\bigg|_{x=\bar{x}}(x_1-\bar{x}_1) \hspace{3.0mm} + \hspace{1.0mm}
    \frac{\partial f_1}{\partial x_2}\bigg|_{x=\bar{x}}(x_2-\bar{x}_2)+ \cancelto{0}{H.O.T.} \\ 
    f_{2, lin}(x) &= \cancelto{0}{f_2(\bar{x})} \hspace{3.0mm} + \hspace{1.0mm}
    \frac{\partial f_2}{\partial x_1}\bigg|_{x=\bar{x}}(x_1-\bar{x}_1) \hspace{3.0mm} + \hspace{1.0mm}
    \frac{\partial f_2}{\partial x_2}\bigg|_{x=\bar{x}}(x_2-\bar{x}_2)+ \cancelto{0}{H.O.T.} \\
    f_{1, lin}(x) &= \frac{\partial f_1}{\partial x_1}\bigg|_{x=\bar{x}}(x_1-\bar{x}_1) \hspace{3.0mm} + \hspace{1.0mm}
    \frac{\partial f_1}{\partial x_2}\bigg|_{x=\bar{x}}(x_2-\bar{x}_2) \\ 
    f_{2, lin}(x) &= \frac{\partial f_2}{\partial x_1}\bigg|_{x=\bar{x}}(x_1-\bar{x}_1) \hspace{3.0mm} + \hspace{1.0mm}
    \frac{\partial f_2}{\partial x_2}\bigg|_{x=\bar{x}}(x_2-\bar{x}_1) \\     
     % \text{\vspace{3.0mm}}
    f_{1, lin}(x) &= x_2 \\
    f_{2, lin}(x) &= -\frac{2(1+x_1^2)^2 - (2x_1)(2(1+x_1^2)(2x_1))}{(1+x_1^2)^4}\bigg|_{(0,0)}x_1 \\
    f_{1, lin}(x) &= x_2 \\
    f_{2, lin}(x) &= -2x_1 \\
    \begin{bmatrix}
      \dot{x_1} \\
      \dot{x_2}
    \end{bmatrix} &=
    \begin{bmatrix}
      -x_1 + x_2 \\
      -x_1 - x_2
    \end{bmatrix} \\
    \begin{bmatrix}
      \dot{x_1} \\
      \dot{x_2}
    \end{bmatrix} &=
    \begin{bmatrix}
     0 & 1 \\
     -2 & 0
    \end{bmatrix} 
    \begin{bmatrix}
      x_1 \\
      x_2
    \end{bmatrix} 
   \end{align*}

  The eigenvalues of the matrix $\begin{bmatrix}
      0 & 1 \\
      -2 & 0
  \end{bmatrix}$ are $\lambda_{1,2} = \pm \sqrt{2}j$
  \newline
  \hangindent=1.58em
  \hangafter=1 Because both of the eignevalues lie along the j$\omega$ axis, we
  know that the linearized system is marginally stable. This tells us nothing
  about the non-linear system though. Instead we will attempt to solve using
  Lyapnuov's 2nd method. \newpage

  We see that $V(x)=\frac{x_1^2}{1+x_1^2}+\frac{1}{2}x_2^2$ is globally positive
  definite. However, the energy function is not radially unbounded. As
  $x_1 -> \infty$ and $x_2$ remains small, V(x) does not go to infinity.
  We will now look at the properties of $\dot{V}(x)$.

  \begin{align*}
    V(x) &= \frac{x_1^2}{1+x_1^2}+\frac{1}{2}x_2^2 \\
    \dot{V}(x) &= \frac{2x_1\dot{x}_1(1+x_1^2)-2x_1\dot{x}_1(x_1^2)}{(1+x_1^2)^2} + x_2\dot{x}_2 \\
    \dot{V}(x) &= \frac{2x_1\dot{x}_1}{(1+x_1^2)^2} + x_2\dot{x}_2 \\
    \dot{V}(x) &= \frac{2x_1x_2}{(1+x_1^2)^2} - \frac{2x_1x_2}{(1+x_1^2)^2} \\
    \dot{V}(x) &= 0
  \end{align*}
  
  $\dot{V}(x)$ is globally negative semi-definite. Since V(x) is
  globally positive definite and unbounded, this leads us to the conclusion that
  \textbf{the system is stable at the equilibrium point (0,0)}.

  (b) Global stability cannot be determined because the Lyapnuov function we
  used in part (a) is not radially unbounded. \newline

  (c) The phase plane is shown below: \newline
  \begin{center}
    \includegraphics[height=75mm]{HW4-3-phaseplane.png} \newline \newline
  \end{center}
  \newpage

  \section{Problem 4}

  Consider the following system
  \begin{align*}
    \dot{x}_1 &= -x_2 + x_1(x_1^2+x_2^2-1) \\
    \dot{x}_2 &= x_1 + x_2(x_1^2+x_2^2-1)
  \end{align*}
  Determine the stability of the origin (0,0) using the following Lyapunov
  function
  \begin{align*}
    V(x) = x_1^2 + x_2^2
  \end{align*}
  Classify the stability in terms of local/global and asymptotic properties

  \vspace{3mm}
  
  \noindent \textit{Solution} \newline \newline

  \noindent V(x) is globally positive definite and radially unbounded. \newline
  To determine the properties of $\dot{V}(x)$, we derive the function.
  \begin{align*}
    \dot{V}(x) = 2x_1\dot{x}_1 + 2x_2\dot{x}_2
  \end{align*}
  \indent Plugging in $\dot{x}_1$ and $\dot{x}_2$
  \begin{align*}
    \dot{V}(x) &= 2x_1(-x_2 + x_1(x_1^2+x_2^2-1)) + 2x_2(x_1 + x_2(x_1^2+x_2^2-1)) \\
    \dot{V}(x) &= 2x_1(\cancelto{0}{-x_2} + x_1(x_1^2+x_2^2-1)) + 2x_2(\cancelto{0}{x_1} + x_2(x_1^2+x_2^2-1)) \\
    \dot{V}(x) &= 2(x_1^2 + x_2^2)(x_1^2+x_2^2-1) 
  \end{align*}

  We can see that $\dot{V}$(x) is locally negative definite for small $x_1,
  x_2$. Therefore, the system is \textbf{locally asymptotically stable} near the
  equilibrium point (0,0).

  \newpage

  \section{Problem 5}

  Examine the stability (local or global, asymptotic or not) of the origin (0,0)
  of the system
  \begin{align*}
    \dot{x}_1 &= x_2 \\
    \dot{x}_2 &= -2x_1 - x_2^3
  \end{align*}
  \begin{enumerate}[label=(\alph*)]
    \item Using a linearization approach
    \item Using a Lyapunov function candidate of the form $V(x_1, x_2) = ax_1^2
    + bx_2^2$.
  \end{enumerate}

  \vspace{3mm}
  
  \noindent \textit{Solution} \newline \newline

   (a)  The formula for linearization of $\dot{x}$ about the point (0,0) is

  \begin{align*}
    f_{1, lin}(x) &= \cancelto{0}{f_1(\bar{x})} \hspace{3.0mm} + \hspace{1.0mm}
    \frac{\partial f_1}{\partial x_1}\bigg|_{(0,0)}(x_1) \hspace{3.0mm} + \hspace{1.0mm}
    \frac{\partial f_1}{\partial x_2}\bigg|_{(0,0)}(x_2)+ \cancelto{0}{H.O.T.} \\ 
    f_{2, lin}(x) &= \cancelto{0}{f_2(\bar{x})} \hspace{3.0mm} + \hspace{1.0mm}
    \frac{\partial f_2}{\partial x_1}\bigg|_{(0,0)}(x_1) \hspace{3.0mm} + \hspace{1.0mm}
    \frac{\partial f_2}{\partial x_2}\bigg|_{(0,0)}(x_2)+ \cancelto{0}{H.O.T.} \\
    f_{1, lin}(x) &= \frac{\partial f_1}{\partial x_1}\bigg|_{(0,0)}(x_1) \hspace{3.0mm} + \hspace{1.0mm}
    \frac{\partial f_1}{\partial x_2}\bigg|_{(0,0)}(x_2) \\ 
    f_{2, lin}(x) &= \frac{\partial f_2}{\partial x_1}\bigg|_{(0,0)}(x_1) \hspace{3.0mm} + \hspace{1.0mm}
    \frac{\partial f_2}{\partial x_2}\bigg|_{(0,0)}(x_2) \\     
     % \text{\vspace{3.0mm}}
    f_{1, lin}(x) &= x_2 \\
    f_{2, lin}(x) &= -2x_1 - 3x_2^2\bigg|_{(0,0)}x_2 \\
    f_{1, lin}(x) &= x_2 \\
    f_{2, lin}(x) &= -2x_1 \\
    \begin{bmatrix}
      \dot{x_1} \\
      \dot{x_2}
    \end{bmatrix} &=
    \begin{bmatrix}
      x_2 \\
      -2x_1
    \end{bmatrix} \\
    \begin{bmatrix}
      \dot{x_1} \\
      \dot{x_2}
    \end{bmatrix} &=
    \begin{bmatrix}
     0 & 1 \\
     -2 & 0
    \end{bmatrix} 
    \begin{bmatrix}
      x_1 \\
      x_2
    \end{bmatrix} 
   \end{align*}

  The eigenvalues of the matrix $\begin{bmatrix}
      0 & 1 \\
      -2 & 0
  \end{bmatrix}$ are $\lambda_{1,2} = \pm \sqrt{2}j$
  \newline
  \hangindent=1.58em
  \hangafter=1 Because both of the eignevalues have negative real parts, we know that \textbf{this
  equilibrium point is marginally stable.} \newpage

  (b) Derive V(x) to find
  \begin{align*}
    \dot{V}(x) = 2ax_1\dot{x}_1 + 2bx_2\dot{x}_2
  \end{align*}
  \indent Plugging in $\dot{x}_1$ and $\dot{x}_2$
  \begin{align*}
    \dot{V}(x) &= 2ax_1x_2 + 2bx_2(-2x_1 - x_2^3) \\
    \dot{V}(x) &= 2ax_1x_2 - 4bx_1x_2 - 2bx_2^4 \\
  \end{align*}

  We can see that if $a=2b$ and $b>0$, the Lyapunov function $V(x_1, x_2) = ax_1^2
    + bx_2^2$ is globally positive definite and radially unbounded. Further, we
    can see that $\dot{V}$(x) is globally negative semi-definite. This, tells us
    that at the equilibrium point (0,0), the system will be \textbf{globally stable}.
 
     
 
\end{document}

