% This is a Latex tutorial document

\documentclass{article}
% \usepackage[letterpaper, landscape, margin=2in]{geometry}
\usepackage[margin=1.0in]{geometry}
\usepackage{amsmath}
\usepackage{amssymb}
\usepackage{empheq}
\usepackage{mathtools}
\usepackage{graphicx}
\usepackage{pgfplots}
\usepackage{cancel}
\usepackage{enumitem}
\graphicspath{ {./../../../Documents/GradSchool/MECE6374/HW5/} }
\makeatletter
\def\@seccntformat#1{%
  \expandafter\ifx\csname c@#1\endcsname\c@section\else
  \csname the#1\endcsname\quad
  \fi}
\makeatother
\title{MECE 6374: Fun Work \#5}
\date{\today}
\author{Eric Eldridge (1561585)}
\begin{document}

  % \setlength{\abovedisplayshortskip}{3pt}
  % \setlength{\belowdisplayshortskip}{3pt}
  % \setlength{\abovedisplayskip}{3pt}
  % \setlength{\belowdisplayskip}{3pt}

  \maketitle

  \section{Problem 1}

  Examine the stability (local or global, asymptotic or not) of the origin (0,0)
  of the system
  \begin{align*}
    \dot{x}_1 &= x_2 \\
    \dot{x}_2 &= -4x_1 -x_2^5
  \end{align*}

  \noindent Using a Lyapunov function candidate of $V(x_1,x_2)=ax_1^2+bx_2^2$
  and LaSalle's Theorem. \newline \newline

  \noindent \textit{Solution} \newline \newline

  We can see that the Lyapunov function candidate, V(x) is globally positive
  definite and radially unbounded. Now, we want to look at the behavior of $\dot{V}(x_1,x_2)$.

  \begin{align*}
    V(x_1,x_2) &= ax_1^2 + bx_2^2 \\
    \dot{V}(x_1,x_2) &= 2ax_1\dot{x}_1 + 2bx_2\dot{x}_2 \\
    \dot{V}(x_1,x_2) &= 2ax_1x_2 + 2bx_2(-4x_1 -x_2^5) \\
    \dot{V}(x_1,x_2) &= 2ax_1x_2 - 8bx_1x_2 -2bx_2^6 
  \end{align*}

  If we let $a = 4b$ s.t. $a,b >0$, then

  \begin{align*}
    \dot{V}(x_1,x_2) &= -2bx_2^6
  \end{align*}

  We can see that $\dot{V}(x_1,x_2)$ is globally negative definite. Since
  $V(x_1,x_2)$ is globally positive definite and radially unbounded, we can say
  that \textbf{the system is globally asymptotically stable at the point (0,0)}.

  \newpage

  \section{Problem 2}

  Consider the following vibrating system with nonlinear damping
  \begin{align*}
    \ddot{y}+d\dot{y}^3+ky=0
  \end{align*}
  where $d,k>0$. Consider the Lyapunov function candidate that corresponds to
  the total system energy
  \begin{align*}
    V(y,\dot{y})=\frac{1}{2}(ky^2+\dot{y}^2)
  \end{align*}
  Is the system locally stable, locally asymptotically stable, globally stable,
  or globally asymptotically stable? Can you use LaSalle's Theorem to to obtain
  stronger stability results? \newline \newline

  \noindent \textit{Solution} \newline \newline

  First, we will put our system in the state-space form. Let $x_1 = y$, $x_2 =
  \dot{y}$. This gives us:
  \begin{align*}
    \dot{x}_1 &= x_2 \\
    \dot{x}_2 &= -kx_1 - dx_2^3
  \end{align*}
  \indent We can also rewrite the Lyapunov candidate function as
  \begin{align*}
    V(x_1,x_2) = \frac{1}{2}(kx_1^2 + x_2^2)
  \end{align*}
  We can see that $V(x_1,x_2)$ is globally positive definite and radially
  unbounded. Now, we want to look at the behavior of $\dot{V}(x_1,x_2)$
  \begin{align*}
    V(x_1,x_2) &= \frac{1}{2}(kx_1^2 + x_2^2) \\
    \dot{V}(x_1,x_2) &= kx_1\dot{x}_1 + x_2\dot{x}_2 \\
    \dot{V}(x_1,x_2) &= \cancelto{}{kx_1x_2} + x_2(\cancelto{}{-kx_1} - dx_2^3) \\
    \dot{V}(x_1,x_2) &= -dx_2^4 \\
  \end{align*}

  This result shows us that $\dot{V}(x_1,x_2) \geq 0$  and is therefore globally positive semi
  definite. Since $V(x_1,x_2)$ is globally positive definite and radially
  unbounded, we can say that our system is globally stable. \newline

  To test if we can further define our system as globally asymptotically stable,
  we turn to LaSalle's Theorem. LaSalle's Theorem states that if a system is
  locally/globally stable AND the set $\dot{V}(\vec{x})=\vec{0}$ contains no
  trajectories other than $\vec{x}=\vec{0}$, then we can say that the system is
  locally/globally asymptotically stable.

  \begin{align*}
    \dot{V}(x_1,x_2) &= -dx_2^4 \\
  \end{align*}

  Therefore, if $\dot{V}=0$, then $x_2=0 \hspace{1mm} \forall \hspace{1mm} t
  \implies \dot{x}_2=0 \hspace{1mm} \forall \hspace{1mm} t$

  \begin{align*}
    \dot{x}_2 = -kx_1 - \cancelto{0}{dx_2^3} = 0 \\
    \implies x_1 = 0
  \end{align*}

  Therefore we have shown that $\dot{V}(\vec{x})=\vec{0}$ only when
  $\vec{x}=\vec{0}$. This satisfies Lasalle's Theorem and we have show that
  \textbf{our system is globally asymptotically stable}.

  \newpage

  \section{Problem 3}

  For the system
  \begin{align*}
    \dot{x}_1 &= -x_1^3 - x_2 \\
    \dot{x}_2 &= x_1^5
  \end{align*}
  \begin{enumerate}[label=(\alph*)]
    \item Find the equilibrium point of the system
    \item Find the linearized system about these points
    \item What stability information can you obtain for the nonlinear system
      from the linearization in part (b)?
    \item Utilize a Lyapunov function $V(x) = \alpha x_1^6 + \beta x_2^2$. What
        stability information do I get?
    \item Can you strengthen the stability assessment in part (d)?
  \end{enumerate} 

  \noindent \textit{Solution} \newline \newline

  (a) The equilibrium points of the system are found by setting
  $\vec{\dot{x}}=\vec{0}$.
  \begin{align*}
    \dot{x}_2 &= x_1^5 = 0 \implies x_1=0 \\
    \dot{x}_1 &= \cancelto{0}{-x_1^3} - x_2 = 0 \implies x_2=0
  \end{align*}

  .\newline \newline

  (b) If we define functions $f_1, f_2$, s.t.
  $\dot{x} =
  \begin{bmatrix}
      \dot{x_1} \\
      \dot{x_2} 
  \end{bmatrix} =
  \begin{bmatrix}
      f_1(x_1, x_2) \\
      f_2(x_1, x_2) \\
  \end{bmatrix} $

  \vspace{5mm} The formula for linearization of $\dot{x}$ is
  \begin{align*}
    f_{i, lin}(x_1,x_2) &= \cancelto{0}{f_i(\bar{x}_1, \bar{x}_2)} \hspace{3.0mm} + \hspace{1.0mm}
    \frac{\partial f_i}{\partial x_1}\bigg|_{(\bar{x}_1, \bar{x}_2)}(x_1-\bar{x}_1) +
    \frac{\partial f_i}{\partial x_2}\bigg|_{(\bar{x}_1, \bar{x}_2}(x_2-\bar{x}_2) \\
    f_{i, lin}(x_1,x_2) &=
    \frac{\partial f_i}{\partial x_1}\bigg|_{(\bar{x}_1, \bar{x}_2)}(x_1-\bar{x}_1) +
    \frac{\partial f_i}{\partial x_2}\bigg|_{(\bar{x}_1, \bar{x}_2)}(x_2-\bar{x}_2) \\
  % \end{align*}
  % \begin{align*}
    f_{1, lin}(x_1,x_2) &=
    \hspace{1.4mm} -3x_1^2\bigg|_{(\bar{x}_1, \bar{x}_1)}(x_1-\bar{x}_1) - 1 \bigg|_{(\bar{x}_1, \bar{x}_1)}(x_2-\bar{x}_2) \\
    f_{2, lin}(x_1,x_2) &= 5x_1^4\bigg|_{(\bar{x}_1, \bar{x}_2)}(x_1-\bar{x}_1) 
     \end{align*}

  \textbf{\underline{Equilibrium Point: (0,\hspace{0.8mm}0)}}

  \begin{align*}
    f_{1, lin}(x_1,x_2) &= \dot{x}_{1,lin} = x_2 \\
    f_{2, lin}(x_1,x_2) &= \dot{x}_{2,lin} =  0 \\
    \begin{bmatrix}
      \dot{x}_{1,lin} \\
      \dot{x}_{2,lin} 
    \end{bmatrix} &=
    \begin{bmatrix}
      0 & 1 \\
      0 & 0
    \end{bmatrix}
    \begin{bmatrix}
      x_1 \\
      x_2 
    \end{bmatrix}
  \end{align*}

  (c) The eigenvalues of the linearized system in part (b) are
  $\lambda_{1,2}=0,0$. This gives us no stability information about the system.

  \newpage (d) $V(x)=\alpha x_1^6 + \beta x_2^2$ is globally positive definite and
  radially unbounded. Now we need to see the properties of $\dot{V}(x)$.
  \begin{align*}
    \dot{V}(x) &= 6\alpha x_1^5\dot{x}_1 + 2\beta x_2\dot{x}_2 \\
    \dot{V}(x) &= 6\alpha x_1^5(-x_1^3 - x_2) + 2\beta x_2(x_1^5) \\
    \dot{V}(x) &= -6\alpha x_1^8 - 6\alpha x_1^5x_2 + 2\beta x_1^5x_2 \\
    Let \hspace{1.0mm} \beta &= 3\alpha \hspace{1.0mm} s.t. \hspace{1.0mm} \alpha , \beta >0 \\
    \dot{V}(x) &= -6\alpha x_1^8 - 6\alpha x_1^5x_2 + 6\alpha x_1^5x_2 \\
    \dot{V}(x) &= -6\alpha x_1^8 
  \end{align*}  

  We can see that $\dot{V}(x)$ is globally negative semi-definite. Since we know
  that V(x) is globally positive definite, we can conclude that \textbf{the
    system is globally stable}. \newline \newline

  (e) LaSalle's Theorem states that for a stable system, if the set of
  x s.t. $\dot{V}(x)=0$ consists strictly of $x=0$ for any time,t, the the system is not only
  stable, but asymptotically stable. \newline
  In part (d) we established that:
  \begin{align*}
    \dot{V}(x) = -6\alpha x_1^8 \\
    \dot{V}(x) = 0 \implies -6\alpha x_1^8 = 0 \\
    \implies x_1 =0 \\
    x_1=0 \hspace{1.0mm} \forall \hspace{1.0mm} t \implies \dot{x}_1=0 \\
    \dot{x}_1 = -x_1^3 - x_2 = 0 \\
    \cancelto{0}{-x_1^3} - x_2 = 0 \\
    \implies x_2 = 0
  \end{align*}

  This shows that $x_1, x_2=0$ for all t and fulfills LaSalle's theorem.
  Therefore we can strengthen our claim from part (d) and say that \textbf{the
    system is globally asymptotically stable}.

  \newpage

  \section{Problem 4}

  Determine which of the following functions of $(x_1, x_2)$ are globally
  positive definite: \newline
  To do this, we put V(x) in the form $V(x)=x^TPx$.
  \begin{enumerate}[label=(\roman*)]
   \item $V(x) = 2x_1^2 + 4x_2^2 - x_1x_2$ \newline \newline
       $V(x) = \begin{bmatrix}
                x_1 & x_2
              \end{bmatrix}
              \begin{bmatrix}
                2 & -\frac{1}{2} \\
                -\frac{1}{2} & 4
              \end{bmatrix}
              \begin{bmatrix}
                x_1 \\
                x_2
              \end{bmatrix}$
      The matrix $\begin{bmatrix}
                2 & -\frac{1}{2} \\
                -\frac{1}{2} & 4
              \end{bmatrix}$ is positive definite \newline \newline
      $\therefore$ \textbf{V(x) is globally positive definite} \newline

    \item $V(x) = 2x_1^2 + 4x_2^2 - 8x_1x_2$ \newline \newline
       $V(x) = \begin{bmatrix}
                x_1 & x_2
              \end{bmatrix}
              \begin{bmatrix}
                2 & -4 \\
                -4 & 4
              \end{bmatrix}
              \begin{bmatrix}
                x_1 \\
                x_2
              \end{bmatrix}$
      The matrix $\begin{bmatrix}
                2 & -4 \\
                -4 & 4
              \end{bmatrix}$ is not positive definite \newline \newline
      $\therefore$ \textbf{V(x) is not globally positive definite} \newline

    \item $V(x) = x_1^2 + x_2^2 - x_1x_2$ \newline \newline
       $V(x) = \begin{bmatrix}
                x_1 & x_2
              \end{bmatrix}
              \begin{bmatrix}
                1 & -\frac{1}{2} \\
                -\frac{1}{2} & 1
              \end{bmatrix}
              \begin{bmatrix}
                x_1 \\
                x_2
              \end{bmatrix}$
      The matrix $\begin{bmatrix}
                1 & -\frac{1}{2} \\
                -\frac{1}{2} & 1
              \end{bmatrix}$ is positive definite \newline \newline
      $\therefore$ \textbf{V(x) is globally positive definite} \newline
 
   \item $V(x) = x_1^2 + x_2^2 - 2x_1x_2$ \newline \newline
       $V(x) = \begin{bmatrix}
                x_1 & x_2
              \end{bmatrix}
              \begin{bmatrix}
                1 & -1 \\
                -1 & 1
              \end{bmatrix}
              \begin{bmatrix}
                x_1 \\
                x_2
              \end{bmatrix}$
      The matrix $\begin{bmatrix}
                1 & -1 \\
                -1 & 1
              \end{bmatrix}$ is not positive definite \newline \newline
      $\therefore$ \textbf{V(x) is not globally positive definite} \newline
 
 
   \item $V(x) = 2x_1^2 - 4x_2^2 + x_1x_2$ \newline \newline
       $V(x) = \begin{bmatrix}
                x_1 & x_2
              \end{bmatrix}
              \begin{bmatrix}
                2 & \frac{1}{2} \\
                \frac{1}{2} & -4 
              \end{bmatrix}
              \begin{bmatrix}
                x_1 \\
                x_2
              \end{bmatrix}$
      The matrix $\begin{bmatrix}
                2 & \frac{1}{2} \\
                \frac{1}{2} & -4
              \end{bmatrix}$ is not positive definite \newline \newline
      $\therefore$ \textbf{V(x) is not globally positive definite} \newline
 
   \item $V(x) = 2x_1^2 + 4x_2^2 + x_1x_2$ \newline \newline
       $V(x) = \begin{bmatrix}
                x_1 & x_2
              \end{bmatrix}
              \begin{bmatrix}
                2 & \frac{1}{2} \\
                \frac{1}{2} & 4 
              \end{bmatrix}
              \begin{bmatrix}
                x_1 \\
                x_2
              \end{bmatrix}$
      The matrix $\begin{bmatrix}
                2 & \frac{1}{2} \\
                \frac{1}{2} & 4
              \end{bmatrix}$ is positive definite \newline \newline
      $\therefore$ \textbf{V(x) is globally positive definite} \newline
 
   \item $V(x) = x_1^2 + x_2^2 + 2x_1x_2$ \newline \newline
       $V(x) = \begin{bmatrix}
                x_1 & x_2
              \end{bmatrix}
              \begin{bmatrix}
                1 & 1 \\
                1 & 1 
              \end{bmatrix}
              \begin{bmatrix}
                x_1 \\
                x_2
              \end{bmatrix}$
      The matrix $\begin{bmatrix}
                1 & 1 \\
                1 & 1
              \end{bmatrix}$ is not positive definite \newline \newline
      $\therefore$ \textbf{V(x) is not globally positive definite} \newline
 
  \end{enumerate}

  \newpage

  \section{Problem 5}

  Consider the linear system
  \begin{align*}
    \dot{x}_1 = Ax
  \end{align*}
  where
  \begin{align*}
    A =
    \begin{bmatrix}
      0 & -1 \\
      1 & -1
    \end{bmatrix}
  \end{align*}
  
  \begin{enumerate}[label=(\alph*)]
    \item Compute the eigenvalues of A and verify that the system is
      asymptotically stable.
    \item Solve the Lyapunov equation
      \begin{align*}
        A^TP + PA + Q = 0
      \end{align*}
      for Q=I and verify that the solution is positive definite.
  \end{enumerate}

  \noindent \textit{Solution} \newline \newline

  (a) The eigenvalues are calculated by setting $|\lambda I-A|=0$

  \begin{align*}
    |\lambda I-A|=0 \\
    det(\begin{bmatrix}
      \lambda & 0 \\
      0 & \lambda
    \end{bmatrix} -
    \begin{bmatrix}
      0 & -1 \\
      1 & -1
    \end{bmatrix}) = 0 \\
    \begin{vmatrix}
      \lambda & 1 \\
      -1 & \lambda + 1
    \end{vmatrix} = 0 \\
    \lambda^2 + \lambda + 1 = 0 \\
    \lambda = -\frac{1}{2} \pm \frac{\sqrt{3}}{2}
  \end{align*}

  Because the eigenvalues of A have negative real parts, we know that the system
  is asymptotically stable. \newline \newline

  (b)
  \begin{align*}
    A^TP + PA + Q = 0 \\
    \begin{bmatrix}
      0 & 1 \\
      -1 &-1
    \end{bmatrix}
    \begin{bmatrix}
      P_{11} & P_{12} \\
      P_{12} & P_{22}
    \end{bmatrix} +
    \begin{bmatrix}
      P_{11} & P_{12} \\
      P_{12} & P_{22}
    \end{bmatrix}
    \begin{bmatrix}
      0 & -1 \\
      1 &-1
    \end{bmatrix}+
    \begin{bmatrix}
      1 & 0 \\
      0 & 1
    \end{bmatrix} =
    \begin{bmatrix}
      0 & 0 \\
      0 & 0
    \end{bmatrix} \\
    \cancelto{0}{0*P_{11}}+1*P_{12} + \cancelto{0}{P_{11}*0}+P_{12}(1) + 1 = 0 \\
    \implies P_{12} = -\frac{1}{2} \\
    \cancelto{0}{0*P_{12}}+1*P_{22} + P_{11}(-1)+P_{12}(-1) + 0 = 0 \\a
    P_{22}+\frac{1}{2} = P_{11} \\
    (-1)*P_{11}+(-1)P_{12} + \cancelto{0}{P_{12}*0}+P_{22}(1) + 0 = 0 \\
    (-1)*P_{12}+(-1)P_{22} + P_{12}(-1)+P_{22}(-1) + 1 = 0\\
    \frac{1}{2}-P_{22} + \frac{1}{2} - P_{22} + 1 = 0 \\
    \implies P_{22}=1 \\
    1+\frac{1}{2} = P_{11} \implies P_{11} = 1.5
  \end{align*}

  We can see $P_{11}=1.5, P_{12}=0.5, P_{22}=1$.
  $P=\begin{bmatrix}
       1.5 & -0.5 \\
       -0.5 & 1
     \end{bmatrix}$

  We can see that this matrix P is positive definite by evaluating that all the
  principal minors are positive.

   

    
 
\end{document}

