% This is a Latex tutorial document

\documentclass{article}
% \usepackage[letterpaper, landscape, margin=2in]{geometry}
\usepackage[margin=1.0in]{geometry}
\usepackage{amsmath}
\usepackage{amssymb}
\usepackage{empheq}
\usepackage{mathtools}
\usepackage{graphicx}
\usepackage{pgfplots}
\usepackage{cancel}
\usepackage{enumitem}
\graphicspath{ {./../../../Documents/GradSchool/MECE6374/HW9/} }
\makeatletter
\def\@seccntformat#1{%
  \expandafter\ifx\csname c@#1\endcsname\c@section\else
  \csname the#1\endcsname\quad
  \fi}
\makeatother
\title{MECE 6374: Fun Work \#9}
\date{\today}
\author{Eric Eldridge (1561585)}
\begin{document}

  % \setlength{\abovedisplayshortskip}{3pt}
  % \setlength{\belowdisplayshortskip}{3pt}
  % \setlength{\abovedisplayskip}{3pt}
  % \setlength{\belowdisplayskip}{3pt}

  \maketitle

  \section{Problem 1}

  For the nonlinear system
  \begin{align*}
    \dot{x}_1 &= -x_1 + x_2 \\
    \dot{x}_2 &= -sin(x_1) +x_1^3 + u
  \end{align*}
  find a feedback control law
  \begin{align*}
    u=u(x_1,x_2)
  \end{align*}
  that results in a feedback linearized system. Design the control law such that
  the closed-loop system is equivalent to a linear system with closed-loop poles
  at -1 and -2.
  \newline \newline

  \noindent \textit{Solution} \newline \newline

  We can see that $\dot{x}_1$ is already linearized and we only need to
  linearize $\dot{x}_2$ which has our control u in it. \newline
  If we choose
  \begin{align*}
    \nonumber u = sin(x_1) - x_1^3 + v
  \end{align*}
  where we will choose $v=Kx$ later when placing our eigenvalues. \newline
  Our system can now be represented as
  \begin{align*}
    \dot{x} = \begin{bmatrix}
      -1 & 1 \\
      0 & 0
    \end{bmatrix}x +
          \begin{bmatrix}
            0 \\
            1
          \end{bmatrix}v
  \end{align*}
  The simplest choice to achieve our desired eigenvalues is $v=-2x_2$. This
  gives the new state space representation of the system as
  \begin{align*}
    \dot{x} = \begin{bmatrix}
      -1 & 1 \\
      0 & -2
    \end{bmatrix}x 
  \end{align*}

  \boldmath
  \begin{empheq}[box=\fbox]{align}
    \nonumber u = sin(x_1) - x_1^3 - 2x_2
  \end{empheq}
  \unboldmath
    
  \newpage

  \section{Problem 2}

  Consider the system:
  \begin{align*}
    \dot{x}_1 &= sin(x_2) \\
    \dot{x}_2 &= cos(x_3) \\
    \dot{x}_3 &= u
  \end{align*}

  \begin{enumerate}[label=(\alph*)]
    \item Show that the following state transformation is a feedback linearizing
      transformation for the system
      \begin{align*}
        z_1 &= x_1 \\
        z_2 &= sin(x_2) \\
        z_3 &= cos(x_2)cos(x_3)
      \end{align*}
    \item Design a state feedback control law so that the closed loop poles of
      the equivalent linearized system are -1, -2 and -3. 
  \end{enumerate}

  \noindent \textit{Solution} \newline \newline

  (a) Taking the derivative of the state transformation gives the following set
  of equations
  
  \begin{align*}
    \dot{z}_1 &= \dot{x}_1 = sin(x_2) \\
    \dot{z}_2 &= \dot{x}_2cos(x_2) = cos(x_2)cos(x_3) \\
    \dot{z}_3 &= -sin(x_2)cos(x_3)\dot{x_2} - cos(x_2)sin(x_3)\dot{x_3} \\
    \dot{z}_3 &= -sin(x_2)cos^2(x_3) + cos(x_2)sin(x_3)u
  \end{align*}
  Next we want to describe the RHS of the equations in terms of $z_1, z_2, z_3$.
  \begin{align*}
    x_1 &= z_1 \\
    x_2 &= sin^{-1}(z_2) \\
    x_3 &= cos^{-1}(\frac{z_3}{cos(sin^{-1}(z_2))}) \\
    x_3 &= cos^{-1}(\frac{z_3}{\sqrt{1-z_2^2})})
  \end{align*}
  
  \begin{align*}
    \dot{z}_1 &= z_2 \\
    \dot{z}_2 &= z_3 \\
    \dot{z}_3 &= -z_2cos^2(x_3) + cos(x_2)sin(x_3)u \\
    \dot{z}_3 &= -\frac{z_2z_3^2}{1-z_2^2} + \cancelto{}{\sqrt{1-z_2^2}}\cdot \frac{\sqrt{1-z_2^2-z_3^2}}{\cancelto{}{\sqrt{1-z_2^2}}}u \\
    \dot{z}_3 &= -\frac{z_2z_3^2}{1-z_2^2} + \sqrt{1-z_2^2-z_3^2}u
  \end{align*}
  We have now linearized the system for every state variable except the one
  containing our control effort, u. We can now choose u s.t. $\dot{z}_3 = v$ 

  \begin{align*}
    \dot{z}_3 &= -\frac{z_2z_3^2}{1-z_2^2} + \sqrt{1-z_2^2-z_3^2}u = v \\
    u &= \frac{v + \frac{z_2z_3^2}{1-z_2^2}}{\sqrt{1-z_2^2-z_3^2}} 
    u = \frac{sin(x_2)cos^2(x_3)}{cos(x_2)sin(x_3)} + v
  \end{align*}
  where we will choose $v=Kz$ later when placing our eigenvalues. \newline
  Our system can now be represented as
  \begin{align*}
    \dot{z} = \begin{bmatrix}
      0 & 1 & 0 \\
      0 & 0 & 1 \\
      0 & 0 & 0 
    \end{bmatrix}z +
          \begin{bmatrix}
            0 \\
            0 \\
            1
          \end{bmatrix}v
  \end{align*}

  We now seek to choose our K (from $v=Kz$) s.t. we have poles at -1, -2, -3.
  Using the place function in Matlab, we find
  \begin{align*}
    K = \begin{bmatrix}
          -6 & -11 & -6
        \end{bmatrix}
  \end{align*}

  We can now use this information to solve for u

  \begin{align*}
    u &= \frac{sin(x_2)cos^2(x_3)}{cos(x_2)sin(x_3)} + v \\
    u &= \frac{sin(x_2)cos^2(x_3)}{cos(x_2)sin(x_3)} - 6z_1 -11z_2 -6z_3 \\
  \end{align*}
   
  \begin{empheq}[box=\fbox]{align}
    \nonumber u &= tan(x_2)\frac{cos^2(x_3)}{sin(x_3)} - 6x_1 -11sin(x_2) -6cos((x_2)cos(x_3)) 
  \end{empheq}
 
  
  \newpage

  \section{Problem 3}

  Consider the system:
  \begin{align*}
    \dot{x}_1 &= e^{x_2} - 1 \\
    \dot{x}_2 &= ax_1^2 + u
  \end{align*}
  
  \begin{enumerate}[label=(\arabic*)]
    \item Is the system feedback linearizable?
    \item Find a nonlinear state transformation that results in a feedback
      linearization of this system
    \item Find the feedback control law that places the closed loop poles of the
      feedback linearized system at -1 and -2.
  \end{enumerate}

  \textit{Solution} \newline \newline
  
  \noindent (1) The NL system $\dot{x}=f(x)+g(x)u$ is feedback linearizable iff
  \begin{enumerate}[label=(\alph*)]
    \item The set
      \begin{align*}
        \{g, ad_fg, ad_f^2g, \cdots, ad_f^{n-1}g\}
      \end{align*}
      is linearly independent
    \item The set
      \begin{align*}
        \{g, ad_fg, ad_f^2g, \cdots, ad_f^{n-2}g\}
      \end{align*}
      is involutive.
  \end{enumerate}

  We can see that $f(x)=\begin{bmatrix}
                          e^{x_2}-1 \\
                          ax_1^2
                        \end{bmatrix}$ and $g(x)=\begin{bmatrix}
                                                    0 \\
                                                    1
                                                  \end{bmatrix}$
                                                  
  \begin{align*}
    ad_fg &= \cancelto{0}{\nabla g} \cdot f - \nabla f \cdot g \\
    ad_fg &= -\begin{bmatrix}
                0 & e^{x_2} \\
                2ax_1 & 0
              \end{bmatrix} \cdot
              \begin{bmatrix}
                0 \\
                1
              \end{bmatrix} \\
    ad_fg &= \begin{bmatrix}
                e^{x_2} \\
                0
              \end{bmatrix}
  \end{align*}
  Now we check the rank of $\begin{bmatrix} 
    g & ad_fg
  \end{bmatrix} =
  \begin{bmatrix}
    0 & e^{x_2} \\
    1 & 0
  \end{bmatrix}$ to verify linear independence. \newline

  \noindent From observation we can see that rank=2 and our first criteria is satisfied.
  \newline \newline

  \noindent For the second criteria, we know that \underline{for a constant vector, g(x),
  involutivity is satisfied.} \newline \newline

  \noindent Therefore, since both criteria are satisfied, we have shown that \textbf{this
    system is feedback linearizable}. \newpage

  (2) Now that we've shown the system is feedback linearizable, we seek to
  find the transformation $z=z(x)$.
  
  \begin{align*}
    \nabla z_1 \cdot g &= 0 \implies
    \begin{bmatrix}
       \frac{\partial z_1}{\partial x_1} & \frac{\partial z_1}{\partial x_2}                                           
    \end{bmatrix}
    \begin{bmatrix}
      0 \\
      1
    \end{bmatrix} = 0 \\
    \implies \frac{\partial z_1}{\partial x_2} &= 0 \\
    \nabla z_1 \cdot ad_fg &\neq 0 \implies
    \begin{bmatrix}
       \frac{\partial z_1}{\partial x_1} & \frac{\partial z_1}{\partial x_2}                                           
    \end{bmatrix}
    \begin{bmatrix}
      e^{x_2} \\
      0
    \end{bmatrix} \neq 0 \\
    \frac{\partial z_1}{\partial x_1} &\neq 0 \\
    Choose \hspace{1mm} z_1 &= x_1 \\
    z_2 &= L_fz_1 = \nabla z_1 \cdot f \\
    z_2 &= \begin{bmatrix}
             1 & 0
           \end{bmatrix} \cdot
           \begin{bmatrix}
             e^{x_2}-1 \\
             ax_1^2
           \end{bmatrix} \\
    z_2 &= e^{x_2} - 1
  \end{align*}
  This gives us the transformation
  \begin{align*}
    z_1 &= x_1 \\
    z_2 &= e^{x_2} - 1 \\
    \dot{z}_1 &= \dot{x}_1 = e^{x_2} - 1 = z_2 \\
    \dot{z}_2 &= e^{x_2}\dot{x}_2 = e^{x_2}(ax_1^2+u) \\
    \dot{z}_2 &= (1+z_2) (az_1^2+u) 
  \end{align*}

  We have now linearized the system for every state variable except the one
  containing our control effort, u. We can now choose u s.t. $\dot{z}_2 = v$ is
  linearized.

  \begin{align*}
    v &= (1+z_2) (az_1^2+u) \\
    u &= \frac{v}{1+z_2} - az_1^2
  \end{align*}
 
  \begin{align*}
    \dot{z} = \begin{bmatrix}
      0 & 1 \\
      0 & 0
    \end{bmatrix}z +
          \begin{bmatrix}
            0 \\
            1
          \end{bmatrix}v
  \end{align*}
  The simplest choice to achieve our desired eigenvalues is $v=Kx$. This
  gives the new state space representation of the system as
  \begin{align*}
    \dot{z} = \begin{bmatrix}
      0 & 1 \\
      k_1 & k_2
    \end{bmatrix}z
  \end{align*}
  (3) To set our eigenvalues at -1,-2, we want the characteristic equation to
  \begin{align*}
    (\lambda + 1)(\lambda + 2) &= \lambda^2 + 3\lambda + 2 = 0
  \end{align*}
  The characteristic equation of our system is
  \begin{align*}
    det(A-\lambda I) &= 0 \\
    (\lambda)(\lambda-k_2) - k_1 &=0 \\
    \lambda^2 - k_2\lambda -k_1 &= 0 = \lambda^2 + 3\lambda + 2 \\
    \implies k_1 = -2, k_2 = -3
  \end{align*}

  Therefore, we have found $z=-2z_1-3z_2$. We can plug this in to find our u that
  will provide feedback linearization.

  \begin{align*}
    u &= \frac{v}{1+z_2} - az_1^2 \\
    u &= \frac{-2z_1 - 3z_2}{1+z_2} - az_1^2 \\
    u &= \frac{-2x_1 - 3(e^{x_2}-1)}{e^{x_2}} - ax_1^2
  \end{align*}
   
  \begin{empheq}[box=\fbox]{align}
    \nonumber u &= \frac{-2x_1 - 3e^{x_2} + 3}{e^{x_2}} - ax_1^2
  \end{empheq}
 
  
  
\end{document}

