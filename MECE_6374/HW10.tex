% This is a Latex tutorial document

\documentclass{article}
% \usepackage[letterpaper, landscape, margin=2in]{geometry}
\usepackage[margin=1.0in]{geometry}
\usepackage{amsmath}
\usepackage{amssymb}
\usepackage{empheq}
\usepackage{mathtools}
\usepackage{graphicx}
\usepackage{pgfplots}
\usepackage{cancel}
\usepackage{enumitem}
\graphicspath{ {./../../../Documents/GradSchool/MECE6374/HW9/} }
\makeatletter
\def\@seccntformat#1{%
  \expandafter\ifx\csname c@#1\endcsname\c@section\else
  \csname the#1\endcsname\quad
  \fi}
\makeatother
\title{MECE 6374: Fun Work \#10}
\date{\today}
\author{Eric Eldridge (1561585)}
\begin{document}

  \maketitle

  \section{Problem 1}

  Consider the following state-space model of an inverted pendulum 
  \begin{align*}
    \dot{x}_1 &= x_2 \\
    \dot{x}_2 &= -\frac{g}{l}sin(x_1 + \frac{\pi}{2}) + \frac{1}{ml^2}u
  \end{align*}
  where $l = 1 m$, $m = 0.1 kg$. The objective is to implement a sliding mode
  control that stabilizes the pendulum at the upright position. 
  \begin{enumerate}[label=(\roman*)]
    \item Show that the stability surface $\sigma (x) = x_1 + x_2$ is a good selection.
    \item Simulate the sliding mode control
      \begin{align*}
        u(x) = -4sgn(x_1 + x_2)
      \end{align*}
    for initial conditions $x_1=1$, $x_2=0$. Plot the states $x_1(t)$ and $x_2(t)$
    as well as the control input u(t).
    \item To eliminate chattering, modify the control law to
      \begin{align*}
        u(x) = -4sat(\frac{\sigma (x)}{\epsilon})
      \end{align*}
    where $\epsilon$ is a small positive scalar. Repeat the closed-loop
    simulations of $x_1(t)$, $x_2(t)$, and $u(t)$.
  \end{enumerate}

  \textbf{\underline{Note:}} sat($\frac{\sigma (x)}{\epsilon}$) is the
  saturation function defined as:

  \[
    \begin{cases} 
      sgn(\sigma (x) & |\sigma (x)| \geq \epsilon \\
      \frac{\sigma (x)}{\epsilon} & |\sigma (x)| \leq \epsilon
    \end{cases}
  \]

  \newpage

  \textit{Solution} \newline \newline

  (i) To show that the stability surface $\sigma (x) = x_1 + x_2$ is a good
  selection we need to show that the surface $\sigma (x) = 0$ is stable.
  \begin{align*}
    x_1 &+ x_2 = 0 \\
    x_2 &= -x_1 \\
    \dot{x_2} &= -\dot{x_1} = -x_2 \\
    \implies x_2(t) &= x_2(0)e^{-t} \\
    \implies x_1(t) &= -x_2(0)e^{-t}
  \end{align*}
  This shows that our stability surface is a good selection. \newline \newline

  (ii) We want to simulate the sliding mode control $u(x) = -4sgn(x_1 + x_2)$
  for our system:
  
  \begin{align*}
    \dot{x}_1 &= x_2 \\
    \dot{x}_2 &= -\frac{g}{l}sin(x_1 + \frac{\pi}{2}) + \frac{1}{ml^2}u
  \end{align*}

  This can be rewritten as

  \begin{align*}
    \dot{x}_1 &= x_2 \\
    \dot{x}_2 &= -\frac{g}{l}sin(x_1 + \frac{\pi}{2}) + \frac{1}{ml^2}(-4sgn(x_1 + x_2)) \\
    \dot{x}_2 &= -\frac{g}{l}cos(x_1) + \frac{1}{ml^2}(-4sgn(x_1 + x_2))
  \end{align*}
   
   
  
\end{document}

