% This is a Latex tutorial document

\documentclass{article}
% \usepackage[letterpaper, landscape, margin=2in]{geometry}
\usepackage[margin=1.0in]{geometry}
\usepackage{amsmath}
\usepackage{amssymb}
\usepackage{empheq}
\usepackage{mathtools}
\usepackage{graphicx}
\usepackage{pgfplots}
\usepackage{cancel}
\usepackage{enumitem}
\graphicspath{ {/Users/ericeldridge/Documents/GradSchool/Fall2018/OptimalControlTheory/HW3/} }
\makeatletter
\def\@seccntformat#1{%
  \expandafter\ifx\csname c@#1\endcsname\c@section\else
  \csname the#1\endcsname\quad
  \fi}
\makeatother
\setlength\parindent{0pt}
\title{MECE 6388: HW \#4}
\date{\today}
\author{Eric Eldridge (1561585)}
\begin{document}

 \maketitle

  \section{3.3-3 Optimal Control of Newton's System}

  Let 
  \begin{align*}
	  \dot{x}_1 &= x_2 \\
	  \dot{x}_2 &= u
  \end{align*}
  have performance index 
  \begin{align*}
	  J=\frac{1}{2}x^T(T)x(T)+\frac{1}{2}\int_0^T(x^Tx+ru^2)dt,
  \end{align*}
  where x = 
  $\begin{bmatrix}
	  x_1 & x_2 
  \end{bmatrix}^T$.
  \begin{enumerate}[label=(\alph*)]
    \item Find the Riccati equation. Write it as three scalar differential equations. Find the feedback gain in terms of the scalar components of S(t).
    \item Write subroutines to find and simulate the optimal control using MATLAB.
    \item Find analytic expressions for the steady-state Riccati solution and gain.
  \end{enumerate}
  
  \ 
  \newline
  \textit{Solution} \newline \newline
  
  a) We can rewrite the state equation as 
  \begin{align*}
	  \dot{x} &= 
	  \begin{bmatrix}
		  0 & 1 \\
		  0 & 0
	  \end{bmatrix}x + 
	  \begin{bmatrix}
		 0 \\
		 1 
	  \end{bmatrix}u
  \end{align*}

  We can see from the state equation and the performance index that A=
  $\begin{bmatrix}
	  0 & 1 \\
	  0 & 0
  \end{bmatrix}$, B=
  $\begin{bmatrix}
         0 \\
         1 
  \end{bmatrix}$, Q=
  $\begin{bmatrix}
	  1 & 0 \\
          0 & 1
  \end{bmatrix}$ \newline

  The Ricatti equation for the given system and parameters is described as
  \begin{align*}
	  -\dot{S}=A^TS+SA-SBR^{-1}B^TS+Q
  \end{align*}
  We can define S as 
  $\begin{bmatrix}
	  s_1 & s_2 \\
          s_2 & s_3
  \end{bmatrix}$.

  Plugging these values into the Ricatti equation gives
  \begin{align*}
	  -\begin{bmatrix}
		  \dot{s_1} & \dot{s_2} \\
		  \dot{s_2} & \dot{s_3}
	  \end{bmatrix} &=
	  \begin{bmatrix}
		  0 & 0 \\
		  1 & 0
	  \end{bmatrix} *
	  \begin{bmatrix}
		  s_1 & s_2 \\
		  s_2 & s_3
	  \end{bmatrix} + 
	  \begin{bmatrix}
		  s_1 & s_2 \\
		  s_2 & s_3
	  \end{bmatrix} *
	  \begin{bmatrix}
		  0 & 1 \\
		  0 & 0
	  \end{bmatrix} -
	  \begin{bmatrix}
		  s_1 & s_2 \\
		  s_2 & s_3
	  \end{bmatrix}
	  \begin{bmatrix}
		  0 \\
		  1
	  \end{bmatrix} * \frac{1}{r} * 
	  \begin{bmatrix}
		  0 & 1
	  \end{bmatrix}
	  \begin{bmatrix}
		  s_1 & s_2 \\
		  s_2 & s_3
	  \end{bmatrix} +
	  \begin{bmatrix}
		  1 & 0 \\
		  0 & 1
	  \end{bmatrix} \\
          &=\begin{bmatrix}
		  0 & 0 \\
		  s_1 & s_2
	  \end{bmatrix} + 
          \begin{bmatrix}
		  0 & s_1 \\
		  0 & s_2
	  \end{bmatrix} - \frac{1}{r}
	  \begin{bmatrix}
		  s_2 \\
		  s_3
	  \end{bmatrix}
	  \begin{bmatrix}
		  s_2 & s_3
	  \end{bmatrix} +
	  \begin{bmatrix}
		  1 & 0 \\
		  0 & 1
	  \end{bmatrix} \\
	  &=\begin{bmatrix}
		  1 & s_1 \\
		  s_1 & 2s_2+1
	  \end{bmatrix} - \frac{1}{r}
	  \begin{bmatrix}
		  s_2^2 & s_2s_3 \\
		  s_2s_3 & s_3^2
	  \end{bmatrix} \\
	  -\begin{bmatrix}
		  \dot{s_1} & \dot{s_2} \\
		  \dot{s_2} & \dot{s_3}
	  \end{bmatrix} &=
	  \begin{bmatrix}
		  1-\frac{1}{r}s_2^2 & s_1-\frac{1}{r}s_2s_3 \\
		  s_1-\frac{1}{r}s_2s_3 & 2s_2-\frac{1}{r}s_3^2+1
	  \end{bmatrix}
  \end{align*}

  \begin{empheq}[box=\fbox]{align}
	  \nonumber \dot{s}_1&=\frac{1}{r}s_2^2-1\\
	  \nonumber \dot{s}_2&=\frac{1}{r}s_2s_3-s_1\\
	  \nonumber \dot{s}_3&=\frac{1}{r}s_3^2-2s_2-1
  \end{empheq}

  The feedback gain can be written as 
  \begin{align*}
	  K&=R^{-1}B^TS \\
	  K&=\frac{1}{r}
	  \begin{bmatrix}
		  0 & 1
	  \end{bmatrix}
	  \begin{bmatrix}
		  s_1 & s_2 \\
		  s_2 & s_3
	  \end{bmatrix} 
  \end{align*}

  \begin{empheq}[box=\fbox]{align}
	  \nonumber K&=\frac{1}{r}
	  \begin{bmatrix}
		  s_2 & s_3
	  \end{bmatrix}
  \end{empheq}

  c) The steady-state Ricatti equation is given by the [], but with $\dot{S}=0$. Therefore
  \begin{align*}
	  0=A^TS+SA-SBR^{-1}B^TS+Q
  \end{align*}
  Plugging in for A, S, B, r, Q
  \begin{align*}
	  0_{2x2}=
	  \begin{bmatrix}
		  1-\frac{1}{r}s_2^2 & s_1-\frac{1}{r}s_2s_3 \\
		  s_1-\frac{1}{r}s_2s_3 & 2s_2-\frac{1}{r}s_3^2+1
	  \end{bmatrix}
  \end{align*}
  This can be written as a series of 3 scalar differential equations with 3 unknown variables ($s_1, s_2, s_3$)
  \begin{align*}
	  0&=\frac{1}{r}s_2^2-1 \\
	  0&=\frac{1}{r}s_2s_3-s_1 \\
	  0&=\frac{1}{r}s_3^2-2s_2-1
  \end{align*}
  Eqn () can be solved for $s_2$
  \begin{align*}
	  0&=\frac{1}{r}s_2^2-1\\
	  s_2&=\pm\sqrt{r} \\
	  s_2&=\sqrt{r} 
  \end{align*}
  We use only the positive value of $s_2$ because S should be positive definite. \newline
  Eqn () can be solved for $s_3$.
  \begin{align*}
	  0&=\frac{1}{r}s_3^2-2s_2-1 \\
	  \frac{1}{r}s_3^2&=2s_2+1 \\
	  s_3&=\sqrt{r(2\sqrt{r}+1)} \\
	  s_3&=\sqrt{2r^{\frac{3}{2}}+r}
  \end{align*}
  Finally, we can use Eqn () to solve for $s_1$.
  \begin{align*}
	  0&=\frac{1}{r}s_2s_3-s_1 \\
	  s_1&=\frac{1}{r}s_2s_3 \\
	  s_1&=\frac{1}{r}\sqrt{r}\sqrt{2r^{\frac{3}{2}}+r} \\
	  s_1&=\sqrt{2r^{\frac{1}{2}}+1} 
  \end{align*}
  Gathering $s_1$, $s_2$, and $s_3$ terms
  \begin{empheq}[box=\fbox]{align}
	  \nonumber s_1&=\sqrt{2r^{\frac{1}{2}}+1} \\
	  \nonumber s_2&=\sqrt{r} \\
	  \nonumber s_3&=\sqrt{2r^{\frac{3}{2}}+r}
  \end{empheq}
  \newpage

  \section{3.3-4 Uncontrolled Newton's System}

  Consider the system of Problem 3.3-3. Solve the Lyapunov equation (3.3-9) to find the cost kernel S(t) if u = 0. Sketch the scalar components of S (t). \newline

  \textit{Solution}
  
  From problem 3.3-3 and the constraint u=0, we have A=
  $\begin{bmatrix}
	  0 & 1 \\
	  0 & 0
  \end{bmatrix}$, B=
  $\begin{bmatrix}
         0 \\
         0 
  \end{bmatrix}$, Q=
  $\begin{bmatrix}
	  1 & 0 \\
          0 & 1
  \end{bmatrix}, S(T)=
  \begin{bmatrix}
	  1 & 0 \\
	  0 & 1
  \end{bmatrix}, r=0, $ \newline
  Again, we can define S as the symmetric matrix
  $\begin{bmatrix}
	  s_1 & s_2 \\
          s_2 & s_3
  \end{bmatrix}$.

  Plugging these values into the steady-state Ricatti equation gives
  \begin{align*}
	  -\begin{bmatrix}
		  \dot{s}_1 & \dot{s}_2 \\
		  \dot{s}_2 & \dot{s}_3 
	  \end{bmatrix} &=
	  \begin{bmatrix}
		  0 & 0 \\
		  1 & 0
	  \end{bmatrix} *
	  \begin{bmatrix}
		  s_1 & s_2 \\
		  s_2 & s_3
	  \end{bmatrix} + 
	  \begin{bmatrix}
		  s_1 & s_2 \\
		  s_2 & s_3
	  \end{bmatrix} *
	  \begin{bmatrix}
		  0 & 1 \\
		  0 & 0
	  \end{bmatrix} +
	  \begin{bmatrix}
		  1 & 0 \\
		  0 & 1
	  \end{bmatrix} \\
          &=\begin{bmatrix}
		  0 & 0 \\
		  s_1 & s_2
	  \end{bmatrix} + 
          \begin{bmatrix}
		  0 & s_1 \\
		  0 & s_2
	  \end{bmatrix} +
	  \begin{bmatrix}
		  1 & 0 \\
		  0 & 1
	  \end{bmatrix} \\
	  &=\begin{bmatrix}
		  1 & s_1 \\
		  s_1 & 2s_2+1
	  \end{bmatrix}
  \end{align*}
  This can be written as a series of 3 scalar equation
  \begin{align*}
	  \dot{s}_1&=-1 \\
	  \dot{s}_2&=-s_1 \\
	  \dot{s}_3&=-2s_2-1
  \end{align*}
  Equation () can be solved for $s_1$
  \begin{align*}
	  \dot{s}_1&=-1 \\
	  s_1(t)&=\int(-1dt) \\
	  s_1(t)&=-t+C_1 \\
	  s_1(T)&=-T+C_1=1 \\
	  \implies C_1&=1+T \\
	  s_1(t)&=-t+1+T
  \end{align*}
  Equation () can be solved for $s_2$
  \begin{align*}
	  \dot{s}_2&=-s_1 \\
	  \dot{s}_2&=t-(1+T) \\
	  s_2(t)&=\int(t-(1+T))dt \\
	  s_2(t)&=\frac{1}{2}t^2-(1+T)t+C_2 \\
	  s_2(T)&=\frac{1}{2}T^2-(1+T)T+C_2=0 \\
	  \implies C_2&=\frac{1}{2}T^2+T \\
	  s_2(t)&=\frac{1}{2}t^2-(1+T)t+\frac{1}{2}T^2+T
  \end{align*}
  Equation () can be solved for $s_3$
  \begin{align*}
	  \dot{s}_3&=-2s_2-1 \\
	  \dot{s}_3&=-t^2+2(1+T)t-T^2-2T-1 \\
	  s_3(t)&=-\frac{1}{3}t^3+(1+T)t^2-(1+2T+T^2)t+C_3 \\
	  s_3(T)&=-\frac{1}{3}T^3+(1+T)T^2-(1+2T+T^2)T+C_3=1 \\
	  \implies C_3&=\frac{1}{3}T^3+T^2+T-1 \\
	  s_3(t)&=-\frac{1}{3}t^3+(1+T)t^2-(1+2T+T^2)t+\frac{1}{3}T^3+T^2+T-1 
  \end{align*}
  Gathering $s_1$, $s_2$, $s_3$ terms
  \begin{empheq}[box=\fbox]{align}
	  \nonumber s_1(t)&=-t+1+T \\
	  \nonumber s_2(t)&=\frac{1}{2}t^2-(1+T)t+\frac{1}{2}T^2+T \\
	  \nonumber s_3(t)&=-\frac{1}{3}t^3+(1+T)t^2-(1+2T+T^2)t+\frac{1}{3}T^3+T^2+T-1 
  \end{empheq}
  \newpage

  \section{3.3-5 Uncontrolled Harmonic Oscillator}

  Repeat problem 3.3-4 for the system in Example 3.3-5. Let S(T) = I, Q = I , $w_n^2 = 1$, $\delta = 0.5$. \newline

  \textit{Solution} \newline \newline

  From example 3.3-5 and the constraint u=0, we have \newline
  A=
  $\begin{bmatrix}
	  0 & 1 \\
	  -\omega_n^2 & -2\delta\omega_n 
  \end{bmatrix}$, B=
  $\begin{bmatrix}
         0 \\
         b 
  \end{bmatrix}$, Q=
  $\begin{bmatrix}
	  1 & 0 \\
          0 & 1
  \end{bmatrix}$, S(T)=
  $\begin{bmatrix}
	  1 & 0 \\
	  0 & 1
  \end{bmatrix}$, r=0 (no input),  \newline
  Again, we can define S as the symmetric matrix
  $\begin{bmatrix}
	  s_1 & s_2 \\
          s_2 & s_3
  \end{bmatrix}$.

  Plugging these values into the steady-state Ricatti equation gives
  \begin{align*}
	  -\begin{bmatrix}
		  \dot{s}_1 & \dot{s}_2 \\
		  \dot{s}_2 & \dot{s}_3 
	  \end{bmatrix} &=
	  \begin{bmatrix}
		  0 & -1 \\
		  1 & 1
	  \end{bmatrix} *
	  \begin{bmatrix}
		  s_1 & s_2 \\
		  s_2 & s_3
	  \end{bmatrix} + 
	  \begin{bmatrix}
		  s_1 & s_2 \\
		  s_2 & s_3
	  \end{bmatrix} *
	  \begin{bmatrix}
		  0 & 1 \\
		  -1 & 1
	  \end{bmatrix} +
	  \begin{bmatrix}
		  1 & 0 \\
		  0 & 1
	  \end{bmatrix} \\
          &=\begin{bmatrix}
		  -s_2 & -s_3 \\
		  s_1-s_2 & s_2-s_3
	  \end{bmatrix} + 
          \begin{bmatrix}
		  -s_2 & s_1-s_2 \\
		  -s_3 & s_2-s_3
	  \end{bmatrix} +
	  \begin{bmatrix}
		  1 & 0 \\
		  0 & 1
	  \end{bmatrix} \\
	  &=\begin{bmatrix}
		  1-s_2 & s_1-s_2-s_3 \\
		  s_1-s_2-s_3 & 2s_2-2s_3+1
	  \end{bmatrix}
  \end{align*}
  This can be written as a series of 3 scalar equations
  \begin{empheq}[box=\fbox]{align}
	  \nonumber \dot{s}_1&=s_2-1 \\
	  \nonumber \dot{s}_2&=s_2+s_3-s_1 \\
	  \nonumber \dot{s}_3&=2s_3-2s_2-1
  \end{empheq}

\end{document}

