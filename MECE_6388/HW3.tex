% This is a Latex tutorial document

\documentclass{article}
% \usepackage[letterpaper, landscape, margin=2in]{geometry}
\usepackage[margin=1.0in]{geometry}
\usepackage{amsmath}
\usepackage{amssymb}
\usepackage{empheq}
\usepackage{mathtools}
\usepackage{graphicx}
\usepackage{pgfplots}
\usepackage{cancel}
\usepackage{enumitem}
\graphicspath{ {/Users/ericeldridge/Documents/GradSchool/Fall2018/OptimalControlTheory/HW3/} }
\makeatletter
\def\@seccntformat#1{%
  \expandafter\ifx\csname c@#1\endcsname\c@section\else
  \csname the#1\endcsname\quad
  \fi}
\makeatother
\title{MECE 6388: HW \#3}
\date{\today}
\author{Eric Eldridge (1561585)}
\begin{document}

 \maketitle

  \section{2.2-7 Cubic performance index}

  Let 
  \begin{align*}
	  x_{k+1}=ax_k+bu_k,
  \end{align*}
  where $x_k$ and $u_k$ are scalars, and 
  \begin{align*}
	  J=\frac{1}{3}s_Nx_N^3+\frac{1}{3}\sum_{k=0}^{N-1}(qx_k^3+ru_k^3)
  \end{align*}
  \begin{enumerate}[label=(\alph*)]
    \item Write state and costate equations and stationarity condition.
    \item When can we solve for $u_k$? Under this condition, eliminate $u_k$ from the equation.
    \item Solve the open-loop control problem (i.e., $x_N$ fixed, $s_N$=0, q=0).
  \end{enumerate}


  \noindent \textit{Solution} \newline \newline
  
  To solve the costate and stationarity conditions, we must first define the Hamiltonian function, $H(x_k,u_k)$
  We start by defining the Hamiltonian function, $H^k$,
  \begin{align*}
    H^k = L^k + \lambda^T_{k+1}f^k
  \end{align*}
  where 
  \begin{align*}
	  L^k &= \frac{1}{3}(qx_k^3+ru_k^3) \\
	  f^k &= ax_k+bu_k
  \end{align*}
  Therefore
  \begin{align*}
	  H^k = \frac{1}{3}(qx_k^3+ru_k^3)+\lambda_{k+1}(ax_k+bu_k)
  \end{align*}

  The state, costate, and stationarity condition are given by the following equations
  \begin{align*}
	  x_{k+1}&=ax_k+bu_k \tag{1.1} \\
	  \lambda_k&=\frac{\partial H^k}{\partial x_k}=qx_k^2+a\lambda_{k+1} \\
	  \lambda_k&=qx_k^2+a\lambda_{k+1} \tag{1.2} \\
	  0&=\frac{\partial H^k}{\partial u_k}=ru_k^2+b\lambda_{k+1} \\
	  u_k^2&=-\frac{b}{r}\lambda_{k+1} \\
	  u_k&=\sqrt{-\frac{b}{r}\lambda_{k+1}} \tag{1.3} 
  \end{align*}

  b) Given the constraints in our equation for $u_k$, we can see that the terms inside the square root must be positive.
  We know that $r>0$ so that gives us the constraint
  \begin{align*}
	  -b\lambda_{k+1} \geq 0
  \end{align*}

  Assuming that this condition is true, we can plug (1.3) into (1.1) and (1.2) to give us the state and costate equations without $u_k$
  \begin{align*}
	  x_{k+1}&=ax_k+b(\sqrt{-\frac{b}{r}\lambda_{k+1}}) \\
  \end{align*}
  \begin{empheq}[box=\fbox]{align}
	  \nonumber x_{k+1}&=ax_k+(\sqrt{-\frac{b^3}{r}\lambda_{k+1}}) \\
	  \nonumber \lambda_k&=qx_k^2+a\lambda_{k+1} 
  \end{empheq}

  c) To solve the open loop control problem, we let $s_N=q=0, r=1$.

  Now we can write our state and costate equations solved in part b) as 
  \begin{align*}
	  \lambda_k&=a\lambda_{k+1} \\
	  \lambda_k&=a^{N-k}\lambda_N \tag{1.4} \\
	  x_{k+1}&=ax_k+(\sqrt{-b^3\lambda_{k+1}}) \\
	  x_{k+1}&=ax_k+\sqrt{-b^3a^{N-k-1}\lambda_N} \tag{1.5}
  \end{align*}
  Plugging in for k=0,1,2,3 gives us
  \begin{align*}
	  x_1&=ax_0+\sqrt{-b^3a^{N-1}\lambda_N} \\
	  x_2&=a^2x_0+\sqrt{-b^3a^{N+1}\lambda_N}+\sqrt{-b^3a^{N-2}\lambda_N} \\
	  x_3&=a^3x_0+\sqrt{-b^3a^{N+3}\lambda_N}+\sqrt{-b^3a^N\lambda_N}+\sqrt{-b^3a^{N-3}\lambda_N} \\
	  x_k&=a^kx_0+\sqrt{-b^3\lambda_N}\sum_{i=0}^{k-1}(\sqrt{a^{N+2k-3-3i}}) \\
	  x_k&=a^kx_0+\sqrt{-b^3\lambda_Na^{N+2k-3}}\sum_{i=0}^{k-1}(a^{-\frac{3}{2}i}) \tag{1.6}
  \end{align*}
  Using the formula for the sum of a geometric series, we can rewrite (1.6) as
  \begin{align*}
	  x_k&=a^kx_0+\sqrt{-b^3\lambda_Na^{N+2k-3}}(\frac{1-a^{-\frac{3}{2}k}}{1-a^{-\frac{3}{2}}}) \tag{1.7}
  \end{align*}
  
  Since the final state is fixed ($x_N=r_N$) we can write
  \begin{align*}
	  &x_N=r_N=a^Nx_0+\sqrt{-b^3\lambda_Na^{3N-3}}(\frac{1-a^{-\frac{3}{2}N}}{1-a^{-\frac{3}{2}}}) \\
	  &\frac{(r_N-a^Nx_0)(1-a^{-\frac{3}{2}})}{1-a^{-\frac{3}{2}N}}=\sqrt{-b^3\lambda_Na^{3N-3}} \\
  \end{align*}
  \begin{empheq}[box=\fbox]{align}
	  \nonumber \lambda_N=-\frac{(r_N-a^Nx_0)^2(1-a^{-\frac{3}{2}})^2}{a^{3N-3}b^3(1-a^{-\frac{3}{2}N})^2} \tag{1.8}
  \end{empheq}

  We can use equation (1.4) to solve for $\lambda_k$ now that we have $\lambda_N$.
  \begin{align*}
	  \lambda_k&=a^{N-k}\lambda_N 
  \end{align*}
  Once we have $\lambda_k$, we can solve for $u_k$ using equation (1.3).
  \begin{align*}
	  u_k&=\sqrt{-\frac{b}{r}\lambda_{k+1}}
  \end{align*}
  Finally, once we have $u_k$, we can solve for $x_k$ forward in time using equation (1.1)
  \begin{align*}
	  x_{k+1}&=ax_k+bu_k 
  \end{align*}


  \newpage

  \section{2.3-1 Digital control of harmonic oscillator}

  A harmonic oscillator is described by
  \begin{align*}
	  \dot{x}_1&=x_2 \tag{2.1} \\
	  \dot{x}_2&=-\omega_n^2x_1+u \tag{2.2}
  \end{align*}
  \begin{enumerate}[label=(\alph*)]
    \item Discretize the plant using a sampling period of T.
    \item With the discretized plant, associate a performance index of 
      \begin{align*}
	      J=\frac{1}{2}[s_1(x_N^1)^2+s_2(x_N^2)^2]+\frac{1}{2}\sum_{k=0}^{N-1}[q_1(x_k^1)^2+q_2(x_k^2)^2+ru_k^2]
      \end{align*}
	  where the state is $x_k=
		  \begin{bmatrix}
			  x_k^1 & x_k^2
		  \end{bmatrix}^T$. Write scalar equations for a digital optimal controller.
    \item Write a MATLAB subroutine to simulate the plant dynamics and use the time response program lsim.m to obtain zero-input state trajectories.
    \item Write a MATLAB subroutine to compute and store the optimal control gains and to update the control $u_k$ given the current state $x_k$. Write a MATLAB driver program to obtain time
	    response plots for the optimal controller.
  \end{enumerate}

  \noindent \textit{Solution}
 
  a) To discretize the plant, we use the following equations.
  \begin{align*}
	  A^s&=e^{AT} \tag{2.3} \\
	  B^s&=\int_0^Te^{A\tau}Bd\tau \tag{2.4}
  \end{align*}
  where A, B are found by putting the plant into the state space form $\dot{x}=Ax+Bu$.
  Equations (2.1) and (2.2) can be rewritten as 
  \begin{align*}
	  \dot{x}=
	  \begin{bmatrix}
		  0 & 1 \\
		  -\omega_n^2 & 0
	  \end{bmatrix}x + 
	  \begin{bmatrix}
		  0 \\
		  1 
	  \end{bmatrix}
  \end{align*}
  which tells us that $A=
  \begin{bmatrix}
	  0 & 1 \\
	  -\omega_n^2 & 0
  \end{bmatrix}$, $B=
  \begin{bmatrix}
         0 \\
         1 
  \end{bmatrix}$.
  We can solve for the eigenvalues of A, $\lambda_{1,2}=\pm\omega_nj$. \newline
  To solve equation (2.3), we will invoke the Cayley-Hamilton Theorem. The Cayley-Hamilton Theorem states that the matrix exponential $e^{AT}$ can be solved by 
  \begin{align*}
	  e^{AT}&=\sum_{k=0}^{n-1}\alpha_kA^k \tag{2.5} \\
	  e^{\lambda_iT}&=\sum_{k=0}^{n-1}\alpha_k\lambda^k \tag{2.6}
  \end{align*}
  where the $\alpha_i$’s are determined from the set of equations given by the eigenvalues of A. \newline
  Using (2.5), we can write 
  \begin{align*}
	  e^{AT}&=\alpha_0I+\alpha_1A
  \end{align*}
  To solve for $\alpha_0, \alpha_1$, we use equation (2.6)
  \begin{align*}
	  e^{-\omega_nTj}&=\alpha_0-j\omega_n\alpha_1 \\
	  cos(-\omega_nT)+jsin(-\omega_nT)&=\alpha_0-j\omega_n\alpha_1 \tag{2.7} \\
	  e^{\omega_nTj}&=\alpha_0+\omega_n\alpha_1j \\
	  cos(\omega_nT)+jsin(\omega_nT)&=\alpha_0+j\omega_n\alpha_1 \tag{2.8}
  \end{align*}
  (2.7)+(2.8) gives us 
  \begin{align*}
	  2cos(\omega_nT)&=2\alpha_0 \\
	  \alpha_0&=cos(\omega_nT) \tag{2.9}
  \end{align*}
  Plugging in (2.9) to (2.7)
  \begin{align*}
	  cos(-\omega_nT)+jsin(-\omega_nT)&=cos(\omega_nT)-j\omega_n\alpha_1 \\
	  sin(-\omega_nT)&=-\omega_n\alpha_1 \\
	  \alpha_1&=\frac{sin(\omega_nT)}{\omega_n} \tag{2.10}
  \end{align*}
  Now that we have $\alpha_0, \alpha_1$, we can find $e^{AT}$ from (2.5)
  \begin{align*}
	  e^{AT}&=(cos(\omega_nT))I+(\frac{sin(\omega_nT)}{\omega_n})A \\
	  e^{AT}&=
	  \begin{bmatrix}
		  cos(\omega_nT) & \frac{sin(\omega_nT)}{\omega_n} \\
		  -\omega_nsin(\omega_nT) & cos(\omega_nT) \tag{2.11}
	  \end{bmatrix}
  \end{align*}

  Now that we have $A^s$, we can plug (2.11) into (2.4) to solve for $B^s$
  \begin{align*}
	  B^s&=\int_0^Te^{A\tau}Bd\tau \\ 
	  B^s&=\int_0^T
	  \begin{bmatrix}
		  cos(\omega_n\tau) & \frac{sin(\omega_n\tau)}{\omega_n} \\
		  -\omega_nsin(\omega_n\tau) & cos(\omega_n\tau)
	  \end{bmatrix}
	  \begin{bmatrix}
		  0 \\
		  1
	  \end{bmatrix}d\tau \\
	  B^s&=\int_0^T
	  \begin{bmatrix}
		  \frac{sin(\omega_n\tau)}{\omega_n} \\
		  cos(\omega_n\tau)
	  \end{bmatrix}d\tau \\
	  B^s&=
	  \begin{bmatrix}
		  \frac{-cos(\omega_n\tau)}{\omega_n^2}\rvert_0^T \\
		  \frac{sin(\omega_n\tau)}{\omega_n}\rvert_0^T
	  \end{bmatrix} \\
	  B^s&=
	  \begin{bmatrix}
		  \frac{1}{\omega_n^2}(1-cos(\omega_nT)) \\
		  \frac{sin(\omega_nT)}{\omega_n} \tag{2.12}
	  \end{bmatrix} 
  \end{align*}
  
  \begin{empheq}[box=\fbox]{align}
	  \nonumber x_{k+1}=
	  \begin{bmatrix}
		  cos(\omega_nT) & \frac{sin(\omega_nT)}{\omega_n} \\
		  -\omega_nsin(\omega_nT) & cos(\omega_nT)
	  \end{bmatrix}x_k + 
	  \begin{bmatrix}
		  \frac{1}{\omega_n^2}(1-cos(\omega_nT)) \\
		  \frac{sin(\omega_nT)}{\omega_n}\
	  \end{bmatrix}u_k
  \end{empheq}

  b) Since we know that $S_k$ is symmetric for all k, let 
  \begin{align*}
	  S_k=\begin{bmatrix}
		  s_1 & s_2 \\
		  s_2 & s_3
	  \end{bmatrix}
  \end{align*}
  Then the feedback gain is updated by
  \begin{align*}
	  \delta&=B^TS_{k+1}B+r \\
	  \delta&=r+\frac{s_1T^4}{4}+s_2T^3+s_3T^2 \\
	  K_k&=B^TS_{k+1}A\frac{1}{\delta}
  \end{align*}
  We can write
  \begin{align*}
	  k_1&=(\frac{s_1T^2}{2}+s_2T)\frac{1}{\delta} \\
	  k_2&+(\frac{s_1T^3}{2}+\frac{3s_2T^2}{2}+s_3T)\frac{1}{\delta}
  \end{align*}
   We can write the closed-loop plant matrix
   \begin{align*}
	   A_k^{cl}&=A-BK_k \\
	   A_k^{cl}&=
	   \begin{bmatrix}
		   a_{11}^{cl} & a_{12}^{cl} \\
		   a_{21}^{cl} & a_{22}^{cl}
	   \end{bmatrix}=
	   \begin{bmatrix}
		   1-\frac{k_1T^a}{2} & T-\frac{k_2T^2}{2} \\
		   -k_1T & 1-k_2T
	   \end{bmatrix}
   \end{align*}
   The updated cost kernel is 
   \begin{align*}
	   S_k&=A^TS_{k+1}A_k^{cl}+Q
   \end{align*}
   which yields the scalar updates:
   \begin{align*}
	   s_1&=s_1a_{11}^{cl}+s_2a_{22}^{cl}+q_d \\
	   s_2&=s_1a_{12}^{cl}+s_2a_{22}^{cl} \\
	   s_3&=(s_1T+s_2)a_{12}^{cl}+(s_2T+s_3)a_{22}^{cl}+q_v
   \end{align*}
  
  c) The unforced system (assuming $w_n=3$, T=0.001) shown below is marginally stable. \newline
  \begin{center}
  \includegraphics[width=8cm]{2-3-1-unforced}
  \end{center}

  d) The response plots for the optimal controller is shown by
  \begin{center}
  \includegraphics[width=8cm]{2-3-1-forced}
  \end{center}

  \newpage The MATLAB code is given below
  \begin{center}
  \includegraphics[width=8cm]{2-3-1-matlab1} \\
  \includegraphics[width=8cm]{2-3-1-matlab2}
  \end{center}


  \newpage

  \section{2.3-2 Digital control of an unstable system}

  Repeat the previous problem for
  \begin{align*}
	  \dot{x}_1&=x_2 \tag{3.1} \\
	  \dot{x}_2&=a^2x_1+bu \tag{3.2}
  \end{align*}

  \noindent \textit{Solution} \newline \newline

  a) To discretize the plant, we use the following equations.
  \begin{align*}
	  A^s&=e^{AT} \tag{3.3} \\
	  B^s&=\int_0^Te^{A\tau}Bd\tau \tag{3.4}
  \end{align*}
  where A, B are found by putting the plant into the state space form $\dot{x}=Ax+Bu$.
  Equations (2.1) and (2.2) can be rewritten as 
  \begin{align*}
	  \dot{x}=
	  \begin{bmatrix}
		  0 & 1 \\
		  a^2 & 0
	  \end{bmatrix}x + 
	  \begin{bmatrix}
		  0 \\
		  b 
	  \end{bmatrix}
  \end{align*}
  which tells us that $A=
  \begin{bmatrix}
	  0 & 1 \\
	  a^2 & 0
  \end{bmatrix}$, $B=
  \begin{bmatrix}
         0 \\
         b 
  \end{bmatrix}$.
  We can solve for the eigenvalues of A, $\lambda_{1,2}=\pm a$. \newline
  To solve equation (3.3), we will invoke the Cayley-Hamilton Theorem. The Cayley-Hamilton Theorem states that the matrix exponential $e^{AT}$ can be solved by 
  \begin{align*}
	  e^{AT}&=\sum_{k=0}^{n-1}\alpha_kA^k \tag{3.5} \\
	  e^{\lambda_iT}&=\sum_{k=0}^{n-1}\alpha_k\lambda^k \tag{3.6}
  \end{align*}
  where the $\alpha_i$’s are determined from the set of equations given by the eigenvalues of A. \newline
  Using (3.5), we can write 
  \begin{align*}
	  e^{AT}&=\alpha_0I+\alpha_1A
  \end{align*}
  To solve for $\alpha_0, \alpha_1$, we use equation (3.6)
  \begin{align*}
	  e^{-aT}&=\alpha_0-a\alpha_1 \tag{3.7} \\
	  e^{aT}&=\alpha_0+a\alpha_1 \tag{3.8} 
  \end{align*}
  (3.7)+(3.8) gives us 
  \begin{align*}
	  (e^{-aT}+e^{aT})&=2\alpha_0 \\
	  \alpha_0&=\frac{1}{2}(e^{-aT}+e^{aT}) \\
	  \alpha_0&=cosh(aT) \tag{3.9}
  \end{align*}
  Plugging in (3.9) to (3.7)
  \begin{align*}
	  e^{-aT}&=\frac{1}{2}(e^{-aT}+e^{aT})-a\alpha_1 \\
	  a\alpha_1&=\frac{1}{2}(e^{aT}-e^{-aT}) \\
	  \alpha_1&=\frac{1}{a}sinh(aT) \tag{3.10}
  \end{align*}
  Now that we have $\alpha_0, \alpha_1$, we can find $e^{AT}$ from (2.5)
  \begin{align*}
	  e^{AT}&=(cosh(aT))I+\frac{1}{a}sinh(aT)A \\
	  e^{AT}&=
	  \begin{bmatrix}
		  cosh(aT) & \frac{1}{a}sinh(aT) \\
		  asinh(aT) & cosh(aT)
	  \end{bmatrix}
  \end{align*}

  Now that we have $A^s$, we can use equation (2.4) to solve for $B^s$
  \begin{align*}
	  B^s&=\int_0^Te^{A\tau}Bd\tau \\ 
	  B^s&=\int_0^T
	  \begin{bmatrix}
		  cosh(a\tau) & \frac{1}{a}sinh(a\tau) \\
		  asinh(a\tau) & cosh(a\tau)
	  \end{bmatrix}
	  \begin{bmatrix}
		  0 \\
		  b
	  \end{bmatrix}d\tau \\
	  B^s&=\int_0^T
	  \begin{bmatrix}
		  \frac{b}{a}sinh(a\tau) \\
		  bcosh(a\tau)
	  \end{bmatrix}d\tau \\
	  B^s&=
	  \begin{bmatrix}
		  \frac{b}{a^2}cosh(a\tau)\rvert_0^T \\
		  \frac{b}{a}sinh(a\tau)\rvert_0^T
	  \end{bmatrix} \\
	  B^s&=
	  \begin{bmatrix}
		  \frac{b}{a^2}(cosh(aT)-1) \\
		  \frac{b}{a}sinh(aT)
	  \end{bmatrix} 
  \end{align*}
  
  \begin{empheq}[box=\fbox]{align}
	  \nonumber x_{k+1}=
	  \begin{bmatrix}
		  cosh(aT) & \frac{1}{a}sinh(aT) \\
		  asinh(aT) & cosh(aT)
	  \end{bmatrix}x_k + 
	  \begin{bmatrix}
		  \frac{b}{a^2}(cosh(aT)-1) \\
		  \frac{b}{a}sinh(aT)
	  \end{bmatrix}u_k
  \end{empheq}
  
  b) Since we know that $S_k$ is symmetric for all k, let 
  \begin{align*}
	  S_k=\begin{bmatrix}
		  s_1 & s_2 \\
		  s_2 & s_3
	  \end{bmatrix}
  \end{align*}
  Then the feedback gain is updated by
  \begin{align*}
	  \delta&=B^TS_{k+1}B+r \\
	  \delta&=r+\frac{s_1T^4}{4}+s_2T^3+s_3T^2 \\
	  K_k&=B^TS_{k+1}A\frac{1}{\delta}
  \end{align*}
  We can write
  \begin{align*}
	  k_1&=(\frac{s_1T^2}{2}+s_2T)\frac{1}{\delta} \\
	  k_2&+(\frac{s_1T^3}{2}+\frac{3s_2T^2}{2}+s_3T)\frac{1}{\delta}
  \end{align*}
   We can write the closed-loop plant matrix
   \begin{align*}
	   A_k^{cl}&=A-BK_k \\
	   A_k^{cl}&=
	   \begin{bmatrix}
		   a_{11}^{cl} & a_{12}^{cl} \\
		   a_{21}^{cl} & a_{22}^{cl}
	   \end{bmatrix}=
	   \begin{bmatrix}
		   1-\frac{k_1T^a}{2} & T-\frac{k_2T^2}{2} \\
		   -k_1T & 1-k_2T
	   \end{bmatrix}
   \end{align*}
   The updated cost kernel is 
   \begin{align*}
	   S_k&=A^TS_{k+1}A_k^{cl}+Q
   \end{align*}
   which yields the scalar updates:
   \begin{align*}
	   s_1&=s_1a_{11}^{cl}+s_2a_{22}^{cl}+q_d \\
	   s_2&=s_1a_{12}^{cl}+s_2a_{22}^{cl} \\
	   s_3&=(s_1T+s_2)a_{12}^{cl}+(s_2T+s_3)a_{22}^{cl}+q_v
   \end{align*}
  The unforced system (assuming a=3, b=2,T=0.001) shown below is clearly unstable. \newline
  \begin{center}
  \includegraphics[width=8cm]{2-3-2-unforced}
  \end{center}

  c) The unforced system (assuming $w_n=3$, T=0.001) shown below is marginally stable. \newline
  \begin{center}
  \includegraphics[width=8cm]{2-3-2-unforced}
  \end{center}

  d) The response plots for the optimal controller is shown by
  \begin{center}
  \includegraphics[width=8cm]{2-3-2-forced}
  \end{center}
  
  The MATLAB code is given below
  \begin{center}
  \includegraphics[width=8cm]{2-3-2-matlab1} \\
  \includegraphics[width=8cm]{2-3-2-matlab2}
  \end{center}
  \newpage

  \section{2.4-1 Steady-state behavior}
  In this problem we consider a rather unrealistic discrete system because it is simple enough to allow an analytic treatment. Thus, let the plant
  \begin{align*}
	  x_{k+1}=
	  \begin{bmatrix}
		  0 & 1 \\
		  0 & 0 
	  \end{bmatrix}x_k+
	  \begin{bmatrix}
		  0 \\
		  1
	  \end{bmatrix}u_k
  \end{align*}
  have the performance index of
  \begin{align*}
	  J_0=\frac{1}{2}s_N^Tx_N+\frac{1}{2}\sum_{k=0}^{N-1}(x_k^T
	  \begin{bmatrix}
		  q_1 & q_2 \\
		  q_2 & q_1
	  \end{bmatrix}x_k+ru_k^2)
  \end{align*}
  \begin{enumerate}[label=(\alph*)]
	  \item Find the optimal steady-state (i.e., N $\rightarrow \infty$) Riccati solution $S_{\infty}^*$ and show that it is positive definite. Find the optimal steady-state gain $K_{\infty}^*$ 
		  and determine when it is nonzero.
	  \item Find the optimal steady-state closed-loop plant and demonstrate its stability.
	  \item Now the suboptimal constant feedback
		  \begin{align*}
			  u_k=-K_{\infty}^*x_k
		  \end{align*}
		  is applied to the plant. Find scalar updates for the components of the sub-optimal cost kernel $S_k$. Find the suboptimal steady-state cost kernel $S_{\infty}$ and demonstrate that 
		  $S_{\infty}=S_{\infty}^*$.
  \end{enumerate}

  \noindent \textit{Solution} \newline \newline
  
  To solve this problem we can analytically solve equation (2.4-12) from the book
  \begin{align*}
	  S&=A^T[S-SB(B^TSB+R)^{-1}B^TS]A+Q
  \end{align*}
  Plugging in A,B,R=r,Q, assuming $S=
  \begin{bmatrix}
	  s_1 & s_2 \\
	  s_3 & s_4
  \end{bmatrix}$
  \begin{align*}
	  S&=A^T[S-SB(
	  \begin{bmatrix}
		  0 & 1
	  \end{bmatrix}
          \begin{bmatrix}
	          s_1 & s_2 \\
	          s_3 & s_4
          \end{bmatrix}
	  \begin{bmatrix}
		  0 \\
		  1
	  \end{bmatrix}+r)^{-1}B^TS]A+Q \\
	  S&=A^T[S-
	  \frac{1}{s_4+r}
	  \begin{bmatrix}
	          s_1 & s_2 \\
	          s_3 & s_4
          \end{bmatrix}
	  \begin{bmatrix}
		  0 \\
		  1
	  \end{bmatrix}
	  \begin{bmatrix}
		  0 & 1
	  \end{bmatrix}
	  \begin{bmatrix}
	          s_1 & s_2 \\
	          s_3 & s_4
          \end{bmatrix}
	  ]A+Q \\
	  S&=
	  \begin{bmatrix}
		  0 & 0 \\
		  1 & 0
	  \end{bmatrix}
	  [\begin{bmatrix}
	          s_1 & s_2 \\
	          s_3 & s_4
          \end{bmatrix}-
	  \frac{1}{s_4+r}
	  \begin{bmatrix}
	          s_2s_3 & s_2s_4 \\
	          s_3s_4 & s_4^2
	  \end{bmatrix}]
	  \begin{bmatrix}
		  0 & 1 \\
		  0 & 0
	  \end{bmatrix}+
	  \begin{bmatrix}
		  q_1 & q_2 \\
		  q_2 & q_1
	  \end{bmatrix} \\
	  \begin{bmatrix}
	          s_1 & s_2 \\
	          s_3 & s_4
	  \end{bmatrix}&=
	  \begin{bmatrix}
		  0 & 0 \\
		  0 & s_1-\frac{s_2s_3}{r+s_4}
	  \end{bmatrix}+
	  \begin{bmatrix}
		  q_1 & q_2 \\
		  q_2 & q_1
	  \end{bmatrix} \\
	  \begin{bmatrix}
	          s_1 & s_2 \\
	          s_3 & s_4
	  \end{bmatrix}&=
	  \begin{bmatrix}
		  q_1 & q_2 \\
		  q_2 & q_1+s_1-\frac{s_2s_3}{r+s_4}
	  \end{bmatrix}
  \end{align*}
  This allows us to solve for $s_i$ in terms of $q_1, q_2$, and r
  \begin{align*}
	  s_1&=q_1 \\
	  s_2&=q_2 \\
	  s_3&=q_2 \\
	  s_4&=q_1+s_1-\frac{s_2s_3}{r+s_4} \\
	  s_4^2+rs_4&=2q_1r+2q_1s_4-s_2s_3 \\
	  s_4^2+rs_4&=2q_1r+2q_1s_4-s_2s_3 \\
	  s_4^2&+(r-2q_1)s_4+(s_2s_3-2q_1r)=0 \\
	  s_4&=\frac{(2q_1-r)\pm \sqrt{(r-2_q1)^2-4(s_2s_3-2q_1r)}}{2}
  \end{align*}
  Now that we have $s_1$, $s_2$, $s_3$, $s_4$, we have
  \begin{empheq}[box=\fbox]{align}
	  \nonumber S_{\infty}=
	  \begin{bmatrix}
		  q_1 & q_2 \\
		  q_2 & s_4
	  \end{bmatrix}
  \end{empheq}
  We can see that $S_{\infty}$ is symmetric and is positive definite if $q_1>0$, $q_1s_4-q_2^2>0$. These are both true so we have shown it is symmetric and positive definite. \newline
  To find $K_{\infty}$ we use the following equation (2.4-13) from the book
  \begin{align*}
	  K_{\infty}&=(B^TS_{\infty}B+R)^{-1}B^TS_{\infty}A \\
	  K_{\infty}&=\frac{1}{r+s_4}
	  \begin{bmatrix}
		  0 & 1
	  \end{bmatrix}S_{\infty}
	  \begin{bmatrix}
		  0 & 1 \\
		  0 & 0 
	  \end{bmatrix}
  \end{align*}
  \begin{empheq}[box=\fbox]{align}
	  \nonumber K_{\infty}&=\frac{1}{r+s_4}
	  \begin{bmatrix}
		  0 & q_2
	  \end{bmatrix}
  \end{empheq}

  b) We now look at evaluating the closed loop system
  \begin{align*}
	  A_{cl}&=A-BK_{\infty} \\
	  A_{cl}&=
	  \begin{bmatrix}
		  0 & 1 \\
		  0 & 0
	  \end{bmatrix}-
	  \begin{bmatrix}
		  0 \\
		  1
	  \end{bmatrix}
	  \begin{bmatrix}
		  0 & q_2
	  \end{bmatrix} \\
	  A_{cl}&=
	  \begin{bmatrix}
		  0 & 1 \\
		  0 & -\frac{q_2}{r+s_4}
	  \end{bmatrix}
  \end{align*}
  This gives us eigevnalues of 
  \begin{empheq}[box=\fbox]{align}
	  \nonumber \lambda_{1,2}=0, -\frac{q_2}{r+s_4}
  \end{empheq}
  Because $q_2$, r, $s_4$ are all positive, we see that both eigenvalues are stable and the closed-loop plant is stable.

  

\newpage
  
  \section{2.4-2 Analytic Riccati solution}
  Let A=$
  \begin{bmatrix}
	  1 & 1 \\
	  0 & 1
  \end{bmatrix}$, B=$
  \begin{bmatrix}
	  0 \\
	  1
  \end{bmatrix}$, $S_N$=I, Q=I

  \begin{enumerate}[label=(\alph*)]
    \item Let r=0.1. Find the Hamiltonian matrix H and its eigenvalues and eigenvectors. Find the analytic expression for Riccati solution $S_k$. Find the steady-state solution
	    $S_{\infty}$ using (2.4-42). FInd the optimal steady-state gain $K_{\infty}$ using (2.4-63) and also using Ackermann's formula.
    \item Let r=1. Find the Hamiltonian matrix and its eigenstructure. Find the steady-state solution $S_{\infty}$ and gain $K_{\infty}$. (Hint: See the discussion following (2.4-63).)
  \end{enumerate}

  \noindent \textit{Solution} \newline \newline

  The Hamiltonian matrix H is given by the following equation
  \begin{align*}
	  H&=\begin{bmatrix}
		  A^{-1} & A^{-1}BR^{-1}B^T \\
		  QA^{-1} & A^T+QA^{-1}BR^{-1}B^T
	  \end{bmatrix} \\
	  H&=\begin{bmatrix}
		  1 & -1 & 0 & -10 \\
		  0 & 1 & 0 & 10 \\
		  1 & -1 & 1 & -10 \\
		  0 & 1 & 1 & 11
	  \end{bmatrix}
  \end{align*}
  We can use MATLAB to determine the eigenvalues and eigenvectors of the matrix H as
  \begin{align*}
	  \lambda_{1,2,3,4}&=10.7802, 2.7654, 0.3616, 0.0928 \\
	  v_{1,2,3,4}&=
	  \begin{bmatrix}
		  -0.1013 \\
		  -0.9907 \\
		  0.0104 \\
		  0.0899
	  \end{bmatrix},
	  \begin{bmatrix}
		  -0.4740 \\
		  -0.8369 \\
		  0.2685 \\
		  0.0534
	  \end{bmatrix},
	  \begin{bmatrix}
		  -0.5081 \\
		  0.3243 \\
		  -0.7959 \\
		  0.0573
	  \end{bmatrix},
	  \begin{bmatrix}
		  -0.5113 \\
		  0.4639 \\
		  -0.5636 \\
		  0.4537
	  \end{bmatrix}
  \end{align*}
  The analytic expression for the Ricatti solution $S_k$ is given by
  \begin{align*}
	  S_k=A^T[S_{k+1}-S_{k+1}B(B^TS_{k+1}B+R)^{-1}B^TS_{k+1})]A+Q
  \end{align*}
  To find the steady-state solution ($S_\infty$), we are instructed to use equation (2.4-42)
  \begin{align*}
	  S_{\infty}=W_{21}W_{11}^{-1}
  \end{align*}
  Therefore to find $S_{\infty}$ we must define and solve W. \newline
  First, let us define M to be a diagonal matrix containing n eigenvalues outside the unit circle. Then we can define D as
  \begin{align*}
	  D&=
	  \begin{bmatrix}
		  M & 0 \\
		  0 & M^{-1}
	  \end{bmatrix} \\
	  D&=
	  \begin{bmatrix}
		  10.7802 & 0 & 0 & 0 \\
		  0 & 2.7654 & 0 & 0 \\
		  0 & 0 & 0.0928 & 0 \\
		  0 & 0 & 0 & 0.3616
	  \end{bmatrix}
  \end{align*}
  W is defined as a nonsingular matrix whose columns are the eigenvectors of H such that
  \begin{align*}
	  &W^{-1}HW=D
  \end{align*}
  Matching our eigenvectors that we found earlier to their eigenvalues in the D matrix, we can find W to be
  \begin{align*}
	  &W=
	  \begin{bmatrix}
		  -0.5113 & -0.5081 & -0.1013 &  0.4740 \\
                   0.4639 &  0.3243 & -0.9907 &  0.8369 \\
                  -0.5636 & -0.7959 &  0.0104 & -0.2685 \\
                   0.4537 &  0.0573 &  0.0899 & -0.0534
	  \end{bmatrix}
  \end{align*}
  Now all that's left to solve for $S_{\infty}$ is to partition W to solve for $W_{11}$, $W_{12}$, $W_{21}$, $W_{22}$.
  \begin{align*}
	  W&=
	  \begin{bmatrix}
		  W_{11} & W_{12} \\
		  W_{21} & W_{22}
	  \end{bmatrix} \\
	  W_{11}&=
	  \begin{bmatrix}
		  -0.5113 & -0.5081 \\
		  0.4639 & 0.3243
	  \end{bmatrix} \\
	  W_{12}&=
	  \begin{bmatrix}
		  -0.1013 & 0.4740 \\
		  -0.9907 & 0.8369
	  \end{bmatrix} \\
	  W_{21}&=\begin{bmatrix}
		  -0.5636 & -0.7959 \\
		  0.4537 & 0.0573
	  \end{bmatrix} \\
	  W_{22}&=\begin{bmatrix}
		  0.0104 & -0.0534 \\
		  0.0899 & -0.0534
	  \end{bmatrix}
  \end{align*}
  We now have everything we need to solve $S_{\infty}$ so we plug $W_{11}$ and $W_{21}$ into equation ()
  \begin{align*}
	  S_{\infty}&=W_{21}W_{11}^{-1} \\
	  S_{\infty}&=
          \begin{bmatrix}
		  -0.5636 & -0.7959 \\
		  0.4537 & 0.0573
	  \end{bmatrix}
	  \begin{bmatrix}
		  -0.5113 & -0.5081 \\
		  0.4639 & 0.3243
	  \end{bmatrix}^{-1} 
  \end{align*}
  \begin{empheq}[box=\fbox]{align}
	  \nonumber S_{\infty}&=
	  \begin{bmatrix}
		  2.6687 & 1.7266 \\
		  1.7266 & 2.8812
	  \end{bmatrix}
  \end{empheq}
  To find $K_{\infty}$, we plug into our equation
  \begin{align*}
	  K_{\infty}&=(B^TS_{\infty}B+R)^{-1}B^TS_{\infty}A \\
  \end{align*}
  \begin{empheq}[box=\fbox]{align}
	  \nonumber K_{\infty}&=
	  \begin{bmatrix}
		  0.5792 & 1.5456
	  \end{bmatrix}
  \end{empheq}

  Ackermann's formula can be used to solve the Algebraic Ricatti Equation (ARE). \newline
  The optimal closed-loop poles are the stable eigenvalues of $H^{-1}$. Using MATLAB, we can calculate these to be
  \begin{align*}
	  \lambda_{1,2,3,4}&=10.7802, 2.7654, 0.3616, 0.0928
  \end{align*}
  Therefore the desired closed loop characteristic polynomial is
  \begin{align*}
	  \Delta^{cl}(z)=z^2-0.4544z+0.0335
  \end{align*}
  The reachability matrix is 
  \begin{align*}
	  U_2&=
	  \begin{bmatrix}
		  B & AB
	  \end{bmatrix} \\
	  U_2&=
	  \begin{bmatrix}
		  0 & 1 \\
		  1 & 1
	  \end{bmatrix}
  \end{align*}
  Finally, we define $e_n$ as the last column of the nxn identity matrix. For our n=2 case
  \begin{align*}
	  e_2=
	  \begin{bmatrix}
		  0 \\
		  1
	  \end{bmatrix}
  \end{align*}
  According to Ackermann's formula
  \begin{align*}
	  K_{\infty}&=e_n^TU_n^{-1}\Delta^d(A) \\
	  K_{\infty}&=
	  \begin{bmatrix}
		  0 & 1 
	  \end{bmatrix}
	  \begin{bmatrix}
		  0 & 1 \\
		  1 & 1
	  \end{bmatrix}^{-1}(A^2-0.4544A+0.0335) 
  \end{align*}
  \begin{empheq}[box=\fbox]{align}
	  \nonumber K_{\infty}&=
	  \begin{bmatrix}
		  0.5792 & 1.5456
	  \end{bmatrix}
  \end{empheq}
  As we can see, these two methods produce the same result as we would expect. \newline \newline

  b) The Hamiltonian matrix H is given by the following equation
  \begin{align*}
	  H&=\begin{bmatrix}
		  A^{-1} & A^{-1}BR^{-1}B^T \\
		  QA^{-1} & A^T+QA^{-1}BR^{-1}B^T
	  \end{bmatrix} \\
	  H&=\begin{bmatrix}
		  1 & -1 & 0 & -1 \\
		  0 & 1 & 0 & 1 \\
		  1 & -1 & 1 & -1 \\
		  0 & 1 & 1 & 2
	  \end{bmatrix}
  \end{align*}
  To find the eigenstructure, we first find the eigenvalues and eigenvectors of $H^{-1}$
  \begin{align*}
	  \lambda_{1,2,3,4}&=2.1220+1.0538i,2.1220-1.0538i,0.3780+0.1877i,0.3780-0.1877i \\
	  v_{1,2,3,4}&=
	  \begin{bmatrix}
		  -0.3330 + 0.3128i \\
		  -0.7032 + 0.0000i \\
		   0.0186 - 0.2962i \\
		   0.4374 + 0.1320i
	  \end{bmatrix},
	  \begin{bmatrix}
		  -0.3330 - 0.3128i \\
		  -0.7032 + 0.0000i \\
		   0.0186 + 0.2962i \\
		   0.4374 - 0.1320i
	  \end{bmatrix},
	  \begin{bmatrix}
		   0.4374 - 0.1320i \\
		  -0.2472 + 0.1642i \\
		   0.7032 + 0.0000i \\
		  -0.1044 + 0.4448i
	  \end{bmatrix},
	  \begin{bmatrix}
		   0.4374 + 0.1320i \\
		  -0.2472 - 0.1642i \\
		   0.7032 + 0.0000i \\
		  -0.1044 - 0.4448i
	  \end{bmatrix}
  \end{align*}
  Now we create the diagonal matrix M, using the stable eigenvalues of $H^{-1}$
  \begin{align*}
	  M=
	  \begin{bmatrix}
		  0.3616 & 0 \\
		  0 & 0.0928
	  \end{bmatrix}
  \end{align*}
  and the associate eigenvectors
  \begin{align*}
	  \begin{bmatrix}
		  X \\
		  \Lambda
	  \end{bmatrix}&=
	  \begin{bmatrix}
		   0.4374 - 0.1320i &  0.4374 + 0.1320i \\
		  -0.2472 + 0.1642i & -0.2472 - 0.1642i \\
		   0.7032 + 0.0000i &  0.7032 + 0.0000i \\
		  -0.1044 + 0.4448i & -0.1044 - 0.4448i
	  \end{bmatrix} \\
	  X&=
	  \begin{bmatrix}
		   0.4374 - 0.1320i &  0.4374 + 0.1320i \\
		  -0.2472 + 0.1642i & -0.2472 - 0.1642i 
	  \end{bmatrix} \\
	  \Lambda&=
	  \begin{bmatrix}
		   0.7032 + 0.0000i &  0.7032 + 0.0000i \\
		  -0.1044 + 0.4448i & -0.1044 - 0.4448i
	  \end{bmatrix} 
  \end{align*}
  Now that we have $\Lambda$ and X, we can plug in to find $S_{\infty}$.
  \begin{align*}
	  S_{\infty}&=\Lambda X^{-1} 
  \end{align*}
  \begin{empheq}[box=\fbox]{align}
	  \nonumber S_{\infty}&=
	  \begin{bmatrix}
		  2.9471 & 2.3692 \\
		  2.3692 & 4.6131
	  \end{bmatrix}
  \end{empheq}
  We can also solve for $K_{\infty}$ using the following equation
  \begin{align*}
	  K_{\infty}&=\frac{1}{r}B^T\Lambda MX^{-1} 
  \end{align*}
  \begin{empheq}[box=\fbox]{align}
	  \nonumber K_{\infty}=
	  \begin{bmatrix}
		  0.4221 & 1.2439
	  \end{bmatrix}
  \end{empheq}


\end{document}

